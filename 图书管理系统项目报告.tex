\documentclass[12pt,a4paper]{article}
\usepackage{xeCJK}
\usepackage{geometry}
\usepackage{graphicx}
\usepackage{hyperref}
\usepackage{listings}
\usepackage{xcolor}
\usepackage{booktabs}
\usepackage{array}
\usepackage{multirow}
\usepackage{longtable}
\usepackage{fancyhdr}
\usepackage{titlesec}
\usepackage{enumitem}
\usepackage{float}
\usepackage{amsmath}
\usepackage{amsfonts}
\usepackage{amssymb}

% 页面设置
\geometry{left=2.5cm,right=2.5cm,top=2.5cm,bottom=2.5cm}

% 中文字体设置 - 使用Overleaf默认中文字体
\setCJKmainfont{FandolSong-Regular}[
  BoldFont = FandolSong-Bold,
  ItalicFont = FandolKai-Regular
]
\setCJKsansfont{FandolHei-Regular}[
  BoldFont = FandolHei-Bold
]
\setCJKmonofont{FandolFang-Regular}

% 页眉页脚设置
\pagestyle{fancy}
\fancyhf{}
\fancyhead[C]{图书管理系统项目报告}
\fancyfoot[C]{\thepage}

% 标题格式设置
\titleformat{\section}{\Large\bfseries}{\thesection}{1em}{}
\titleformat{\subsection}{\large\bfseries}{\thesubsection}{1em}{}
\titleformat{\subsubsection}{\normalsize\bfseries}{\thesubsubsection}{1em}{}

% 代码块设置
\lstset{
    basicstyle=\ttfamily\small,
    keywordstyle=\color{blue}\bfseries,
    commentstyle=\color{green!60!black},
    stringstyle=\color{red},
    showstringspaces=false,
    breaklines=true,
    frame=single,
    backgroundcolor=\color{gray!10},
    numbers=left,
    numberstyle=\tiny\color{gray},
    stepnumber=1,
    numbersep=10pt,
    tabsize=4,
    extendedchars=true,
    inputencoding=utf8,
    escapeinside={(*@}{@*)}
}

% 超链接设置
\hypersetup{
    colorlinks=true,
    linkcolor=black,
    urlcolor=blue,
    citecolor=blue
}

\title{\textbf{\Huge 图书管理系统\\数据库设计与实现报告}}
\author{}
\date{}

\begin{document}

\maketitle
\thispagestyle{empty}

\vspace{2cm}

\begin{center}
\Large
\textbf{基于MySQL数据库的图书管理系统设计与性能优化研究}

\vspace{1cm}

\normalsize
课程名称:数据库系统原理\\
指导教师:XXX教授\\
完成时间:\today

\vspace{3cm}

\begin{table}[H]
\centering
\begin{tabular}{|c|c|c|c|}
\hline
\textbf{角色} & \textbf{姓名} & \textbf{学号} & \textbf{主要分工} \\
\hline
组长 & 冯达 & 072108208 & 数据库设计、后端开发、系统架构 \\
\hline
组员A & 张三 & 123123 & 数据库优化、索引设计、前端开发 \\
\hline
组员B & 李四 & 4564546 & 数据库建模、存储过程、API接口 \\
\hline
组员C & 王五 & 789789 & 数据库测试、性能调优、文档编写 \\
\hline
\end{tabular}
\end{table}

\end{center}

\newpage
\tableofcontents
\newpage

\section{项目概述}

\subsection{项目背景}
随着信息技术的快速发展和数据量的爆炸式增长,传统的图书管理方式已经无法满足现代图书馆的管理需求。图书馆作为知识管理的重要场所,涉及复杂的数据关系管理,包括用户信息、图书资源、借阅记录、分类体系等多维度数据。

为了深入理解和实践数据库系统原理,我们选择图书管理系统作为研究对象,通过设计和实现一个完整的图书管理数据库系统,探索关系型数据库的设计原则、规范化理论、事务管理、并发控制、索引优化等核心概念。

本项目采用MySQL作为核心数据库管理系统,结合Spring Boot框架的数据访问层技术,构建了一个功能完善、性能优化的图书管理系统,旨在验证数据库系统理论在实际应用中的有效性。

\subsection{项目目标}

\subsubsection{数据库系统目标}
\begin{itemize}
    \item \textbf{数据库设计目标}:运用规范化理论设计满足至少3NF的关系模式
    \item \textbf{数据完整性目标}:实现实体完整性、参照完整性和用户定义完整性约束
    \item \textbf{并发控制目标}:设计支持多用户并发访问的事务处理机制
    \item \textbf{性能优化目标}:通过索引设计和查询优化提升系统响应性能
    \item \textbf{安全管理目标}:实现数据访问控制和权限管理机制
\end{itemize}

\subsubsection{应用系统目标}
\begin{itemize}
    \item 实现图书信息的数字化管理和高效检索
    \item 提供完整的图书借阅业务流程管理
    \item 支持多角色用户的权限控制和数据隔离
    \item 提供实时的数据统计和分析功能
    \item 确保系统的高可用性和数据一致性
\end{itemize}

\subsection{项目特色}
\begin{itemize}
    \item \textbf{前后端分离架构}:采用现代化的前后端分离设计,提高系统可维护性
    \item \textbf{响应式设计}:支持多种设备访问,适配PC端和移动端
    \item \textbf{角色权限控制}:支持管理员和读者两种角色,权限分离明确
    \item \textbf{实时数据更新}:基于RESTful API,实现前后端数据实时同步
    \item \textbf{安全认证机制}:采用JWT令牌认证,确保系统安全
\end{itemize}

\section{需求分析}

\subsection{功能性需求}

\subsubsection{管理员功能需求}
\begin{enumerate}
    \item \textbf{系统管理}
    \begin{itemize}
        \item 用户账户管理(增删改查)
        \item 权限分配和管理
        \item 系统参数配置
    \end{itemize}
    
    \item \textbf{图书管理}
    \begin{itemize}
        \item 图书信息录入、修改、删除
        \item 图书分类管理
        \item 图书库存管理
        \item 图书搜索和查询
    \end{itemize}
    
    \item \textbf{借阅管理}
    \begin{itemize}
        \item 借阅申请审核
        \item 图书归还处理
        \item 逾期图书管理
        \item 借阅统计和报表
    \end{itemize}
    
    \item \textbf{数据统计}
    \begin{itemize}
        \item 图书借阅统计
        \item 用户活跃度统计
        \item 系统使用情况分析
    \end{itemize}
\end{enumerate}

\subsubsection{读者功能需求}
\begin{enumerate}
    \item \textbf{账户管理}
    \begin{itemize}
        \item 用户注册和登录
        \item 个人信息维护
        \item 密码修改
    \end{itemize}
    
    \item \textbf{图书浏览}
    \begin{itemize}
        \item 图书搜索和筛选
        \item 图书详情查看
        \item 图书分类浏览
    \end{itemize}
    
    \item \textbf{借阅服务}
    \begin{itemize}
        \item 在线借阅申请
        \item 借阅记录查询
        \item 借阅申请取消
        \item 借阅到期提醒
    \end{itemize}
\end{enumerate}

\subsection{非功能性需求}

\subsubsection{性能需求}
\begin{itemize}
    \item 系统响应时间不超过3秒
    \item 支持并发用户数不少于100人
    \item 数据库查询效率优化
    \item 前端页面加载速度优化
\end{itemize}

\subsubsection{安全需求}
\begin{itemize}
    \item 用户身份认证和授权
    \item 数据传输加密
    \item SQL注入防护
    \item XSS攻击防护
    \item 敏感数据保护
\end{itemize}

\subsubsection{可用性需求}
\begin{itemize}
    \item 系统可用性达到99\%以上
    \item 支持7×24小时运行
    \item 具备故障恢复能力
    \item 支持数据备份和恢复
\end{itemize}

\section{数据库系统分析与设计}

\subsection{数据库需求分析}

\subsubsection{业务规则分析}
图书管理系统涉及以下核心业务规则:

\begin{enumerate}
    \item \textbf{用户管理规则}
    \begin{itemize}
        \item 每个用户具有唯一的用户名和ID
        \item 用户分为管理员(ADMIN)和读者(READER)两种角色
        \item 用户状态可以是启用或禁用状态
        \item 每个用户必须有真实姓名,邮箱可选但唯一
    \end{itemize}
    
    \item \textbf{图书管理规则}
    \begin{itemize}
        \item 每本图书具有唯一的ID,ISBN可选但必须唯一
        \item 图书必须属于某个分类,分类可以包含多本图书
        \item 图书有总数量和可借数量两个属性
        \item 图书状态包括可借阅、不可借阅、维护中
    \end{itemize}
    
    \item \textbf{借阅管理规则}
    \begin{itemize}
        \item 一个用户可以借阅多本图书,一本图书可以被多个用户借阅
        \item 借阅记录包含申请、批准、借出、归还、逾期、拒绝等状态
        \item 借阅操作必须在图书可借数量大于0时才能进行
        \item 用户最多同时借阅5本图书
        \item 借阅期限为30天,超期自动标记为逾期
    \end{itemize}
    
    \item \textbf{数据完整性规则}
    \begin{itemize}
        \item 删除用户时,其借阅记录应保留(设置为NULL或保留历史)
        \item 删除图书时,必须确保没有未归还的借阅记录
        \item 借阅记录的状态变更必须遵循业务流程
        \item 图书的可借数量不能超过总数量
    \end{itemize}
\end{enumerate}

\subsubsection{数据对象识别}
通过业务分析,识别出以下主要数据对象:

\begin{itemize}
    \item \textbf{用户(User)}:系统使用者,包括管理员和读者
    \item \textbf{图书(Book)}:图书馆的图书资源
    \item \textbf{分类(Category)}:图书的分类信息
    \item \textbf{借阅记录(BorrowRecord)}:用户借阅图书的记录
\end{itemize}

\subsection{概念模型设计}

\subsubsection{E-R图设计}
基于数据对象分析,设计实体-关系图:

\begin{figure}[H]
\centering
\begin{tabular}{|c|c|c|}
\hline
\multicolumn{3}{|c|}{\textbf{实体及其属性}} \\
\hline
\textbf{用户(User)} & \textbf{图书(Book)} & \textbf{分类(Category)} \\
\hline
ID (主键) & ID (主键) & ID (主键) \\
用户名 (唯一) & 标题 & 名称 (唯一) \\
密码 & 作者 & 描述 \\
真实姓名 & ISBN (唯一) & 创建时间 \\
邮箱 (唯一) & 出版社 & 更新时间 \\
电话 & 出版日期 & \\
角色 & 描述 & \\
状态 & 总数量 & \\
创建时间 & 可借数量 & \\
更新时间 & 价格 & \\
& 状态 & \\
& 创建时间 & \\
& 更新时间 & \\
\hline
\end{tabular}
\caption{主要实体及属性}
\end{figure}

\textbf{关系设计:}
\begin{itemize}
    \item \textbf{分类-图书关系}:一对多关系,一个分类可以包含多本图书
    \item \textbf{用户-借阅记录关系}:一对多关系,一个用户可以有多条借阅记录
    \item \textbf{图书-借阅记录关系}:一对多关系,一本图书可以有多条借阅记录
    \item \textbf{用户-图书关系}:多对多关系,通过借阅记录实现
\end{itemize}

\subsubsection{关系模式设计}
将E-R图转换为关系模式:

\textbf{用户关系:}\\
User(\underline{id}, username, password, real\_name, email, phone, role, enabled, created\_at, updated\_at)

\textbf{分类关系:}\\
Category(\underline{id}, name, description, created\_at, updated\_at)

\textbf{图书关系:}\\
Book(\underline{id}, title, author, isbn, publisher, publish\_date, description, cover\_image, total\_quantity, available\_quantity, price, \textit{category\_id}, status, created\_at, updated\_at)

\textbf{借阅记录关系:}\\
BorrowRecord(\underline{id}, \textit{user\_id}, \textit{book\_id}, status, borrow\_date, due\_date, return\_date, remarks, created\_at, updated\_at)

其中,\underline{下划线}表示主键,\textit{斜体}表示外键。

\subsection{规范化分析}

\subsubsection{第一范式(1NF)分析}
所有关系都满足第一范式要求:
\begin{itemize}
    \item 每个属性都是原子性的,不可再分
    \item 每个属性值都是单一数据类型
    \item 不存在重复的属性组
\end{itemize}

\subsubsection{第二范式(2NF)分析}
检查是否存在部分函数依赖:
\begin{itemize}
    \item \textbf{User关系}:主键为id,所有非主属性完全依赖于id
    \item \textbf{Category关系}:主键为id,所有非主属性完全依赖于id
    \item \textbf{Book关系}:主键为id,所有非主属性完全依赖于id
    \item \textbf{BorrowRecord关系}:主键为id,所有非主属性完全依赖于id
\end{itemize}

所有关系都满足第二范式要求,不存在部分函数依赖。

\subsubsection{第三范式(3NF)分析}
检查是否存在传递函数依赖:

\textbf{潜在的传递依赖分析:}
\begin{itemize}
    \item \textbf{Book关系中的分类信息}:category\_id → category\_name
    \begin{itemize}
        \item 原设计将分类信息独立为Category表,通过外键关联
        \item 避免了传递依赖,满足3NF要求
    \end{itemize}
    
    \item \textbf{BorrowRecord关系中的用户和图书信息}:
    \begin{itemize}
        \item user\_id → user\_name (传递依赖)
        \item book\_id → book\_title (传递依赖)
        \item 通过外键引用独立的User和Book表,避免数据冗余
        \item 满足3NF要求
    \end{itemize}
\end{itemize}

\textbf{规范化结论:}所有关系模式都满足第三范式(3NF)要求,有效避免了数据冗余和更新异常。

\section{数据库物理设计}

\subsection{存储结构设计}

\subsubsection{表空间和存储引擎选择}
\begin{itemize}
    \item \textbf{存储引擎}:选择InnoDB引擎
    \begin{itemize}
        \item 支持事务处理(ACID特性)
        \item 支持外键约束
        \item 支持行级锁定,提高并发性能
        \item 支持崩溃恢复
    \end{itemize}
    
    \item \textbf{字符集}:使用utf8mb4字符集
    \begin{itemize}
        \item 支持完整的UTF-8编码
        \item 兼容中文、英文及特殊字符
        \item 排序规则使用utf8mb4\_unicode\_ci
    \end{itemize}
\end{itemize}

\subsubsection{数据类型优化}
\begin{itemize}
    \item \textbf{主键设计}:使用BIGINT AUTO\_INCREMENT
    \begin{itemize}
        \item 支持大数据量的ID分配
        \item 自动递增,避免ID冲突
        \item 占用8字节,性能优良
    \end{itemize}
    
    \item \textbf{字符串类型选择}:
    \begin{itemize}
        \item 短字符串:VARCHAR(20-100),节省存储空间
        \item 长文本:TEXT类型用于描述等长内容
        \item 固定长度:对于状态等枚举值使用ENUM类型
    \end{itemize}
    
    \item \textbf{时间类型}:使用TIMESTAMP
    \begin{itemize}
        \item 自动时区转换
        \item 支持自动更新(ON UPDATE CURRENT\_TIMESTAMP)
        \item 范围广,精度高
    \end{itemize}
\end{itemize}

\subsection{索引设计}

\subsubsection{主键索引}
所有表的主键自动创建聚簇索引:
\begin{itemize}
    \item \textbf{users表}:PRIMARY KEY (id)
    \item \textbf{categories表}:PRIMARY KEY (id)  
    \item \textbf{books表}:PRIMARY KEY (id)
    \item \textbf{borrow\_records表}:PRIMARY KEY (id)
\end{itemize}

\subsubsection{唯一索引}
为保证数据唯一性创建的索引:
\begin{lstlisting}[language=sql]
-- 用户表唯一索引
CREATE UNIQUE INDEX uk_users_username ON users(username);
CREATE UNIQUE INDEX uk_users_email ON users(email);

-- 分类表唯一索引  
CREATE UNIQUE INDEX uk_categories_name ON categories(name);

-- 图书表唯一索引
CREATE UNIQUE INDEX uk_books_isbn ON books(isbn);
\end{lstlisting}

\subsubsection{外键索引}
为外键字段创建索引以提高连接查询性能:
\begin{lstlisting}[language=sql]
-- 图书表外键索引
CREATE INDEX idx_books_category_id ON books(category_id);

-- 借阅记录表外键索引
CREATE INDEX idx_borrow_records_user_id ON borrow_records(user_id);
CREATE INDEX idx_borrow_records_book_id ON borrow_records(book_id);
\end{lstlisting}

\subsubsection{业务查询索引}
根据业务查询需求创建的复合索引:
\begin{lstlisting}[language=sql]
-- 图书搜索复合索引
CREATE INDEX idx_books_search ON books(title, author, publisher);

-- 借阅状态查询索引
CREATE INDEX idx_borrow_records_status ON borrow_records(status);

-- 用户借阅记录查询索引
CREATE INDEX idx_borrow_records_user_status 
ON borrow_records(user_id, status, created_at);

-- 图书借阅统计索引
CREATE INDEX idx_borrow_records_book_date 
ON borrow_records(book_id, borrow_date, return_date);

-- 逾期查询索引
CREATE INDEX idx_borrow_records_due_date 
ON borrow_records(due_date, status);
\end{lstlisting}

\subsection{完整性约束设计}

\subsubsection{实体完整性约束}
通过主键约束保证实体完整性:
\begin{lstlisting}[language=sql]
-- 主键约束自动保证实体完整性
ALTER TABLE users ADD CONSTRAINT pk_users PRIMARY KEY (id);
ALTER TABLE categories ADD CONSTRAINT pk_categories PRIMARY KEY (id);
ALTER TABLE books ADD CONSTRAINT pk_books PRIMARY KEY (id);
ALTER TABLE borrow_records ADD CONSTRAINT pk_borrow_records PRIMARY KEY (id);
\end{lstlisting}

\subsubsection{参照完整性约束}
通过外键约束保证参照完整性:
\begin{lstlisting}[language=sql]
-- 图书分类外键约束
ALTER TABLE books 
ADD CONSTRAINT fk_books_category_id 
FOREIGN KEY (category_id) REFERENCES categories(id) 
ON DELETE RESTRICT ON UPDATE CASCADE;

-- 借阅记录用户外键约束
ALTER TABLE borrow_records 
ADD CONSTRAINT fk_borrow_records_user_id 
FOREIGN KEY (user_id) REFERENCES users(id) 
ON DELETE SET NULL ON UPDATE CASCADE;

-- 借阅记录图书外键约束
ALTER TABLE borrow_records 
ADD CONSTRAINT fk_borrow_records_book_id 
FOREIGN KEY (book_id) REFERENCES books(id) 
ON DELETE RESTRICT ON UPDATE CASCADE;
\end{lstlisting}

\subsubsection{用户定义完整性约束}
通过检查约束和触发器保证业务规则:
\begin{lstlisting}[language=sql]
-- 图书数量约束
ALTER TABLE books 
ADD CONSTRAINT chk_books_quantity 
CHECK (available_quantity >= 0 AND available_quantity <= total_quantity);

-- 用户角色约束
ALTER TABLE users 
ADD CONSTRAINT chk_users_role 
CHECK (role IN ('ADMIN', 'READER'));

-- 借阅状态约束
ALTER TABLE borrow_records 
ADD CONSTRAINT chk_borrow_records_status 
CHECK (status IN ('PENDING', 'APPROVED', 'BORROWED', 'RETURNED', 'OVERDUE', 'REJECTED'));

-- 借阅日期逻辑约束
ALTER TABLE borrow_records 
ADD CONSTRAINT chk_borrow_records_dates 
CHECK (due_date > borrow_date AND 
       (return_date IS NULL OR return_date >= borrow_date));
\end{lstlisting}

\subsection{存储过程和触发器设计}

\subsubsection{图书借阅存储过程}
\begin{lstlisting}[language=sql]
DELIMITER //
CREATE PROCEDURE sp_BorrowBook(
    IN p_user_id BIGINT,
    IN p_book_id BIGINT,
    OUT p_result VARCHAR(100)
)
BEGIN
    DECLARE book_available INT DEFAULT 0;
    DECLARE user_borrow_count INT DEFAULT 0;
    DECLARE EXIT HANDLER FOR SQLEXCEPTION
    BEGIN
        ROLLBACK;
        SET p_result = 'Error: Transaction failed';
    END;

    START TRANSACTION;
    
    -- 检查图书可借数量
    SELECT available_quantity INTO book_available 
    FROM books WHERE id = p_book_id FOR UPDATE;
    
    -- 检查用户当前借阅数量
    SELECT COUNT(*) INTO user_borrow_count 
    FROM borrow_records 
    WHERE user_id = p_user_id AND status IN ('PENDING', 'APPROVED', 'BORROWED');
    
    -- 业务规则检查
    IF book_available <= 0 THEN
        SET p_result = 'Error: Book not available';
        ROLLBACK;
    ELSEIF user_borrow_count >= 5 THEN
        SET p_result = 'Error: User borrow limit exceeded';
        ROLLBACK;
    ELSE
        -- 插入借阅记录
        INSERT INTO borrow_records (user_id, book_id, status, created_at)
        VALUES (p_user_id, p_book_id, 'PENDING', NOW());
        
        SET p_result = 'Success: Borrow request created';
        COMMIT;
    END IF;
END //
DELIMITER ;
\end{lstlisting}

\subsubsection{图书数量更新触发器}
\begin{lstlisting}[language=sql]
-- 借阅状态更新时自动调整图书可借数量
DELIMITER //
CREATE TRIGGER tr_UpdateBookQuantity
AFTER UPDATE ON borrow_records
FOR EACH ROW
BEGIN
    -- 当状态从PENDING/APPROVED变为BORROWED时,减少可借数量
    IF OLD.status IN ('PENDING', 'APPROVED') AND NEW.status = 'BORROWED' THEN
        UPDATE books 
        SET available_quantity = available_quantity - 1 
        WHERE id = NEW.book_id;
    END IF;
    
    -- 当状态变为RETURNED时,增加可借数量
    IF OLD.status = 'BORROWED' AND NEW.status = 'RETURNED' THEN
        UPDATE books 
        SET available_quantity = available_quantity + 1 
        WHERE id = NEW.book_id;
    END IF;
END //
DELIMITER ;
\end{lstlisting}

\subsubsection{逾期检查触发器}
\begin{lstlisting}[language=sql]
-- 定期检查逾期记录的事件调度器
DELIMITER //
CREATE EVENT ev_CheckOverdueBooks
ON SCHEDULE EVERY 1 DAY
STARTS CURRENT_TIMESTAMP
DO
BEGIN
    UPDATE borrow_records 
    SET status = 'OVERDUE'
    WHERE status = 'BORROWED' 
    AND due_date < CURDATE()
    AND return_date IS NULL;
END //
DELIMITER ;
\end{lstlisting}

\section{系统设计}

\subsection{系统架构}

本系统采用前后端分离的架构设计,主要包含以下几个层次:

\begin{figure}[H]
\centering
\begin{tabular}{|c|}
\hline
\textbf{前端展示层 (Vue 3 + Element Plus)} \\
\hline
用户界面 | 交互逻辑 | 状态管理 \\
\hline
\hline
\textbf{API接口层 (RESTful API)} \\
\hline
HTTP请求/响应 | JSON数据格式 | 状态码 \\
\hline
\hline
\textbf{业务逻辑层 (Spring Boot)} \\
\hline
控制器 | 服务层 | 业务逻辑处理 \\
\hline
\hline
\textbf{数据访问层 (Spring Data JPA)} \\
\hline
实体映射 | 数据库操作 | 事务管理 \\
\hline
\hline
\textbf{数据存储层 (MySQL)} \\
\hline
数据持久化 | 数据完整性 | 索引优化 \\
\hline
\end{tabular}
\caption{系统架构图}
\end{figure}

\subsection{技术栈}

\subsubsection{后端技术栈}
\begin{itemize}
    \item \textbf{Spring Boot 3.5.0}:主框架,提供依赖注入、自动配置等功能
    \item \textbf{Spring Boot Web}:Web MVC框架,构建RESTful API
    \item \textbf{Spring Boot Data JPA}:数据访问层,简化数据库操作
    \item \textbf{Spring Boot Security}:安全框架,处理认证和授权
    \item \textbf{Spring Boot Validation}:数据验证框架
    \item \textbf{MySQL Connector/J}:MySQL数据库驱动
    \item \textbf{JWT 0.12.3}:JSON Web Token实现 (jjwt-api, jjwt-impl, jjwt-jackson)
    \item \textbf{Lombok}:简化Java代码编写,减少样板代码
    \item \textbf{Maven 3.6+}:项目构建和依赖管理工具
    \item \textbf{Java 17}:开发语言版本
\end{itemize}

\subsubsection{前端技术栈}
\begin{itemize}
    \item \textbf{Vue 3.5.13}:渐进式JavaScript框架
    \item \textbf{Element Plus 2.8.8}:基于Vue 3的UI组件库
    \item \textbf{Pinia 2.3.0}:Vue 3的状态管理库
    \item \textbf{Vue Router 4.5.0}:Vue.js官方路由管理器
    \item \textbf{Axios 1.7.9}:基于Promise的HTTP客户端
    \item \textbf{Vite 6.3.5}:快速的前端构建工具
    \item \textbf{@vueuse/core}:Vue组合式API工具集
    \item \textbf{dayjs}:轻量级日期处理库
\end{itemize}

\subsection{数据库设计}

\subsubsection{数据库概念模型}
系统主要包含以下实体及其关系:
\begin{itemize}
    \item \textbf{用户(User)}:系统用户信息
    \item \textbf{图书(Book)}:图书基本信息
    \item \textbf{分类(Category)}:图书分类信息
    \item \textbf{借阅记录(BorrowRecord)}:借阅相关信息
\end{itemize}

\subsubsection{数据表设计}

\textbf{1. 用户表(users)}
\begin{longtable}{|p{2.5cm}|p{2.5cm}|p{4.5cm}|p{2.5cm}|}
\hline
\textbf{字段名} & \textbf{类型} & \textbf{约束} & \textbf{说明} \\
\hline
id & BIGINT & PRIMARY KEY, AUTO\_INCREMENT & 用户ID \\
\hline
username & VARCHAR(20) & UNIQUE, NOT NULL & 用户名 \\
\hline
password & VARCHAR(255) & NOT NULL & 密码(加密) \\
\hline
real\_name & VARCHAR(50) & NOT NULL & 真实姓名 \\
\hline
email & VARCHAR(100) & UNIQUE & 邮箱地址 \\
\hline
phone & VARCHAR(20) & - & 电话号码 \\
\hline
role & ENUM & NOT NULL, DEFAULT 'READER' & 用户角色 \\
\hline
enabled & BOOLEAN & DEFAULT TRUE & 是否启用 \\
\hline
created\_at & TIMESTAMP & DEFAULT CURRENT\_TIMESTAMP & 创建时间 \\
\hline
updated\_at & TIMESTAMP & DEFAULT CURRENT\_TIMESTAMP ON UPDATE & 更新时间 \\
\hline
\end{longtable}

\textbf{2. 图书表(books)}
\begin{longtable}{|p{2.5cm}|p{2.5cm}|p{4.5cm}|p{2.5cm}|}
\hline
\textbf{字段名} & \textbf{类型} & \textbf{约束} & \textbf{说明} \\
\hline
id & BIGINT & PRIMARY KEY, AUTO\_INCREMENT & 图书ID \\
\hline
title & VARCHAR(200) & NOT NULL & 书名 \\
\hline
author & VARCHAR(100) & NOT NULL & 作者 \\
\hline
isbn & VARCHAR(20) & UNIQUE & ISBN号 \\
\hline
publisher & VARCHAR(100) & - & 出版社 \\
\hline
publish\_date & VARCHAR(20) & - & 出版日期 \\
\hline
description & TEXT & - & 图书描述 \\
\hline
cover\_image & VARCHAR(500) & - & 封面图片URL \\
\hline
category\_id & BIGINT & FOREIGN KEY & 分类ID \\
\hline
total\_quantity & INT & NOT NULL, DEFAULT 0 & 总数量 \\
\hline
available\_quantity & INT & NOT NULL, DEFAULT 0 & 可借数量 \\
\hline
price & DECIMAL(10,2) & - & 图书价格 \\
\hline
status & ENUM & DEFAULT 'AVAILABLE' & 图书状态 \\
\hline
created\_at & TIMESTAMP & DEFAULT CURRENT\_TIMESTAMP & 创建时间 \\
\hline
updated\_at & TIMESTAMP & DEFAULT CURRENT\_TIMESTAMP ON UPDATE & 更新时间 \\
\hline
\end{longtable}

\textbf{3. 分类表(categories)}
\begin{longtable}{|p{2.5cm}|p{2.5cm}|p{4.5cm}|p{2.5cm}|}
\hline
\textbf{字段名} & \textbf{类型} & \textbf{约束} & \textbf{说明} \\
\hline
id & BIGINT & PRIMARY KEY, AUTO\_INCREMENT & 分类ID \\
\hline
name & VARCHAR(50) & UNIQUE, NOT NULL & 分类名称 \\
\hline
description & VARCHAR(200) & - & 分类描述 \\
\hline
created\_at & TIMESTAMP & DEFAULT CURRENT\_TIMESTAMP & 创建时间 \\
\hline
updated\_at & TIMESTAMP & DEFAULT CURRENT\_TIMESTAMP ON UPDATE & 更新时间 \\
\hline
\end{longtable}

\textbf{4. 借阅记录表(borrow\_records)}
\begin{longtable}{|p{2.5cm}|p{2.5cm}|p{4.5cm}|p{2.5cm}|}
\hline
\textbf{字段名} & \textbf{类型} & \textbf{约束} & \textbf{说明} \\
\hline
id & BIGINT & PRIMARY KEY, AUTO\_INCREMENT & 记录ID \\
\hline
user\_id & BIGINT & FOREIGN KEY, NOT NULL & 用户ID \\
\hline
book\_id & BIGINT & FOREIGN KEY, NOT NULL & 图书ID \\
\hline
status & ENUM & NOT NULL, DEFAULT 'PENDING' & 借阅状态 \\
\hline
borrow\_date & TIMESTAMP & - & 借阅日期 \\
\hline
due\_date & TIMESTAMP & - & 应还日期 \\
\hline
return\_date & TIMESTAMP & - & 实际归还日期 \\
\hline
remarks & VARCHAR(500) & - & 备注信息 \\
\hline
created\_at & TIMESTAMP & DEFAULT CURRENT\_TIMESTAMP & 创建时间 \\
\hline
updated\_at & TIMESTAMP & DEFAULT CURRENT\_TIMESTAMP ON UPDATE & 更新时间 \\
\hline
\end{longtable}

\textbf{ENUM字段详细说明:}
\begin{itemize}
    \item \textbf{用户角色(role)}:'ADMIN'(管理员), 'READER'(读者)
    \item \textbf{图书状态(status)}:'AVAILABLE'(可借阅), 'UNAVAILABLE'(不可借阅), 'MAINTENANCE'(维护中)
    \item \textbf{借阅状态(status)}:'PENDING'(待审核), 'APPROVED'(已批准), 'BORROWED'(已借出), 'RETURNED'(已归还), 'OVERDUE'(已逾期), 'REJECTED'(已拒绝)
\end{itemize}

\section{系统实现}

\subsection{数据库实体映射实现}

\subsubsection{JPA实体类设计}
基于数据库设计,使用JPA注解实现对象关系映射:

\textbf{1. User实体类}
\begin{lstlisting}[language=java]
@Entity
@Table(name = "users")
@Data
@NoArgsConstructor
@AllArgsConstructor
public class User {
    
    @Id
    @GeneratedValue(strategy = GenerationType.IDENTITY)
    private Long id;
    
    @NotBlank(message = "用户名不能为空")
    @Size(min = 3, max = 20, message = "用户名长度必须在3-20个字符之间")
    @Column(unique = true, nullable = false)
    private String username;
    
    @NotBlank(message = "密码不能为空")
    @Size(min = 6, message = "密码长度至少6个字符")
    @Column(nullable = false)
    private String password;
    
    @NotBlank(message = "真实姓名不能为空")
    @Column(nullable = false)
    private String realName;
    
    @Email(message = "邮箱格式不正确")
    @Column(unique = true)
    private String email;
    
    @Enumerated(EnumType.STRING)
    @Column(nullable = false)
    private Role role = Role.READER;
    
    @Column(nullable = false)
    private Boolean enabled = true;
    
    @CreationTimestamp
    @Column(nullable = false, updatable = false)
    private LocalDateTime createdAt;
    
    @UpdateTimestamp
    @Column(nullable = false)
    private LocalDateTime updatedAt;
    
    @OneToMany(mappedBy = "user", cascade = CascadeType.ALL, fetch = FetchType.LAZY)
    private List<BorrowRecord> borrowRecords;
    
    public enum Role {
        ADMIN("管理员"), READER("读者");
        private final String description;
        Role(String description) { this.description = description; }
        public String getDescription() { return description; }
    }
}
\end{lstlisting}

\textbf{2. Book实体类}
\begin{lstlisting}[language=java]
@Entity
@Table(name = "books")
@Data
@NoArgsConstructor
@AllArgsConstructor
public class Book {
    
    @Id
    @GeneratedValue(strategy = GenerationType.IDENTITY)
    private Long id;
    
    @NotBlank(message = "图书标题不能为空")
    @Column(nullable = false)
    private String title;
    
    @NotBlank(message = "作者不能为空")
    @Column(nullable = false)
    private String author;
    
    @Column(unique = true)
    private String isbn;
    
    @Column(columnDefinition = "TEXT")
    private String description;
    
    @Column
    private String coverImage;
    
    @NotNull(message = "总数量不能为空")
    @PositiveOrZero(message = "总数量不能为负数")
    @Column(nullable = false)
    private Integer totalQuantity = 0;
    
    @NotNull(message = "可借数量不能为空")
    @PositiveOrZero(message = "可借数量不能为负数")
    @Column(nullable = false)
    private Integer availableQuantity = 0;
    
    @Column(precision = 10, scale = 2)
    private BigDecimal price;
    
    @ManyToOne(fetch = FetchType.LAZY)
    @JoinColumn(name = "category_id")
    private Category category;
    
    @Enumerated(EnumType.STRING)
    @Column(nullable = false)
    private Status status = Status.AVAILABLE;
    
    @CreationTimestamp
    @Column(nullable = false, updatable = false)
    private LocalDateTime createdAt;
    
    @UpdateTimestamp
    @Column(nullable = false)
    private LocalDateTime updatedAt;
    
    @OneToMany(mappedBy = "book", cascade = CascadeType.ALL, fetch = FetchType.LAZY)
    private List<BorrowRecord> borrowRecords;
    
    public enum Status {
        AVAILABLE("可借阅"), UNAVAILABLE("不可借阅"), MAINTENANCE("维护中");
        private final String description;
        Status(String description) { this.description = description; }
        public String getDescription() { return description; }
    }
}
\end{lstlisting}

\textbf{3. BorrowRecord实体类}
\begin{lstlisting}[language=java]
@Entity
@Table(name = "borrow_records")
@Data
@NoArgsConstructor
@AllArgsConstructor
public class BorrowRecord {
    
    @Id
    @GeneratedValue(strategy = GenerationType.IDENTITY)
    private Long id;
    
    @ManyToOne(fetch = FetchType.LAZY)
    @JoinColumn(name = "user_id", nullable = false)
    @NotNull(message = "用户不能为空")
    private User user;
    
    @ManyToOne(fetch = FetchType.LAZY)
    @JoinColumn(name = "book_id", nullable = false)
    @NotNull(message = "图书不能为空")
    private Book book;
    
    @Enumerated(EnumType.STRING)
    @Column(nullable = false)
    private Status status = Status.PENDING;
    
    @Column
    private LocalDateTime borrowDate;
    
    @Column
    private LocalDateTime dueDate;
    
    @Column
    private LocalDateTime returnDate;
    
    @CreationTimestamp
    @Column(nullable = false, updatable = false)
    private LocalDateTime createdAt;
    
    @UpdateTimestamp
    @Column(nullable = false)
    private LocalDateTime updatedAt;
    
    public enum Status {
        PENDING("待审核"), APPROVED("已批准"), BORROWED("已借出"),
        RETURNED("已归还"), OVERDUE("已逾期"), REJECTED("已拒绝");
        private final String description;
        Status(String description) { this.description = description; }
        public String getDescription() { return description; }
    }
}
\end{lstlisting}

\subsection{后端实现}

\subsubsection{项目结构}
\begin{lstlisting}[language=bash]
library/
├── src/
│   ├── main/
│   │   ├── java/com/example/library/
│   │   │   ├── LibraryApplication.java    # 主启动类
│   │   │   ├── config/                    # 配置类
│   │   │   │   ├── SecurityConfig.java   # 安全配置
│   │   │   │   ├── GlobalExceptionHandler.java # 全局异常处理
│   │   │   │   └── DataInitializer.java  # 数据初始化
│   │   │   ├── controller/                # 控制器层
│   │   │   │   ├── AuthController.java   # 认证控制器
│   │   │   │   ├── AdminController.java  # 管理员控制器
│   │   │   │   ├── ReaderController.java # 读者控制器
│   │   │   │   ├── BookController.java   # 图书控制器
│   │   │   │   └── TestController.java   # 测试控制器
│   │   │   ├── dto/                       # 数据传输对象
│   │   │   ├── entity/                    # 实体类
│   │   │   │   ├── User.java             # 用户实体
│   │   │   │   ├── Book.java             # 图书实体
│   │   │   │   ├── Category.java         # 分类实体
│   │   │   │   └── BorrowRecord.java     # 借阅记录实体
│   │   │   ├── repository/                # 数据访问层
│   │   │   ├── service/                   # 服务层
│   │   │   ├── filter/                    # 过滤器
│   │   │   │   └── JwtAuthenticationFilter.java
│   │   │   └── util/                      # 工具类
│   │   │       └── JwtUtil.java          # JWT工具类
│   │   └── resources/
│   │       └── application.properties     # 应用配置
│   └── test/                              # 测试代码
├── library-ui/                            # 前端项目
│   ├── src/
│   │   ├── App.vue                       # 主应用组件
│   │   ├── main.js                       # 入口文件
│   │   ├── style.css                     # 全局样式
│   │   ├── components/                   # 公共组件
│   │   ├── views/                        # 页面组件
│   │   │   ├── Login.vue                 # 登录页面
│   │   │   ├── Dashboard.vue             # 仪表盘
│   │   │   ├── Test.vue                  # 测试页面
│   │   │   ├── admin/                    # 管理员页面
│   │   │   │   ├── BookManagement.vue   # 图书管理
│   │   │   │   ├── CategoryManagement.vue # 分类管理
│   │   │   │   ├── BorrowManagement.vue # 借阅管理
│   │   │   │   └── UserManagement.vue   # 用户管理
│   │   │   └── reader/                   # 读者页面
│   │   │       ├── BookList.vue          # 图书列表
│   │   │       ├── MyBorrows.vue         # 我的借阅
│   │   │       └── Profile.vue           # 个人资料
│   │   ├── layouts/                      # 布局组件
│   │   │   └── Layout.vue               # 主布局
│   │   ├── router/                       # 路由配置
│   │   ├── stores/                       # 状态管理
│   │   │   └── auth.js                  # 认证状态
│   │   └── utils/                        # 工具函数
│   │       └── api.js                   # API封装
│   ├── package.json                      # 依赖配置
│   ├── vite.config.js                   # Vite配置
│   └── index.html                       # HTML模板
├── pom.xml                               # Maven配置
├── database_init.sql                     # 数据库初始化脚本
└── README.md                            # 项目说明
\end{lstlisting}

\subsubsection{核心实现}

\textbf{1. 安全配置}
\begin{lstlisting}[language=java]
@Configuration
@EnableWebSecurity
public class SecurityConfig {
    
    @Bean
    public SecurityFilterChain filterChain(HttpSecurity http) throws Exception {
        http.csrf(csrf -> csrf.disable())
            .cors(cors -> cors.configurationSource(corsConfigurationSource()))
            .sessionManagement(session -> 
                session.sessionCreationPolicy(SessionCreationPolicy.STATELESS))
            .authorizeHttpRequests(auth -> auth
                .requestMatchers("/api/auth/**").permitAll()
                .requestMatchers("/api/admin/**").hasRole("ADMIN")
                .requestMatchers("/api/reader/**").hasAnyRole("READER", "ADMIN")
                .anyRequest().authenticated()
            )
            .addFilterBefore(jwtAuthenticationFilter, 
                UsernamePasswordAuthenticationFilter.class);
        
        return http.build();
    }
}
\end{lstlisting}

\textbf{2. JWT工具类}
\begin{lstlisting}[language=java]
@Component
public class JwtUtil {
    
    @Value("${jwt.secret}")
    private String secret;
    
    @Value("${jwt.expiration}")
    private Long expiration;
    
    private SecretKey getSigningKey() {
        return Keys.hmacShaKeyFor(secret.getBytes());
    }
    
    public String generateToken(String username, String role) {
        Map<String, Object> claims = new HashMap<>();
        claims.put("role", role);
        return createToken(claims, username);
    }
    
    private String createToken(Map<String, Object> claims, String subject) {
        return Jwts.builder()
                .claims(claims)
                .subject(subject)
                .issuedAt(new Date(System.currentTimeMillis()))
                .expiration(new Date(System.currentTimeMillis() + expiration))
                .signWith(getSigningKey())
                .compact();
    }
    
    public String getUsernameFromToken(String token) {
        return getClaimFromToken(token, Claims::getSubject);
    }
    
    public String getRoleFromToken(String token) {
        return getClaimFromToken(token, claims -> (String) claims.get("role"));
    }
    
    public Boolean validateToken(String token) {
        try {
            Jwts.parser()
                .verifyWith(getSigningKey())
                .build()
                .parseSignedClaims(token);
            return true;
        } catch (JwtException | IllegalArgumentException e) {
            return false;
        }
    }
}
\end{lstlisting}

\subsection{前端实现}

\subsubsection{项目结构}
\begin{lstlisting}[language=bash]
library-ui/src/
├── components/          # 公共组件
├── views/              # 页面组件
│   ├── admin/          # 管理员页面
│   │   ├── BookManagement.vue
│   │   ├── CategoryManagement.vue
│   │   ├── BorrowManagement.vue
│   │   └── UserManagement.vue
│   ├── reader/         # 读者页面
│   │   ├── BookList.vue
│   │   ├── MyBorrows.vue
│   │   └── Profile.vue
│   ├── Dashboard.vue   # 仪表盘
│   ├── Login.vue       # 登录页面
│   └── Test.vue        # 测试页面
├── layouts/            # 布局组件
│   └── Layout.vue      # 主布局
├── router/             # 路由配置
├── stores/             # 状态管理
├── utils/              # 工具函数
└── style.css          # 全局样式
\end{lstlisting}

\subsubsection{核心实现}

\textbf{1. 状态管理 (Composition API)}
\begin{lstlisting}[language=javascript]
import { defineStore } from 'pinia'
import { ref, computed } from 'vue'
import { authApi } from '@/utils/api'

export const useAuthStore = defineStore('auth', () => {
  // 状态
  const token = ref(localStorage.getItem('token') || '')
  const user = ref(JSON.parse(localStorage.getItem('user') || 'null'))
  const loading = ref(false)

  // 计算属性
  const isAuthenticated = computed(() => !!token.value && !!user.value)
  const isAdmin = computed(() => user.value?.role === 'ADMIN')
  const isReader = computed(() => user.value?.role === 'READER')

  // 动作
  const login = async (credentials) => {
    loading.value = true
    try {
      const response = await authApi.login(credentials)
      const { token: newToken, user: userData } = response.data
      
      token.value = newToken
      user.value = userData
      
      // 保存到本地存储
      localStorage.setItem('token', newToken)
      localStorage.setItem('user', JSON.stringify(userData))
      
      return response
    } finally {
      loading.value = false
    }
  }

  const logout = () => {
    token.value = ''
    user.value = null
    localStorage.removeItem('token')
    localStorage.removeItem('user')
  }

  return {
    // 状态和计算属性
    token, user, loading, isAuthenticated, isAdmin, isReader,
    // 动作
    login, logout
  }
})
\end{lstlisting}

\textbf{2. API请求封装}
\begin{lstlisting}[language=javascript]
import axios from 'axios'
import { ElMessage } from 'element-plus'

// 创建axios实例
const request = axios.create({
  baseURL: '/api',
  timeout: 10000
})

// 请求拦截器
request.interceptors.request.use(
  (config) => {
    const token = localStorage.getItem('token')
    if (token) {
      config.headers.Authorization = `Bearer ${token}`
    }
    return config
  },
  (error) => Promise.reject(error)
)

// 响应拦截器
request.interceptors.response.use(
  (response) => {
    const { success, message, data } = response.data
    if (success) {
      return { data, message }
    } else {
      ElMessage.error(message || '请求失败')
      return Promise.reject(new Error(message || '请求失败'))
    }
  },
  (error) => {
    let errorMessage = '网络错误'
    
    if (error.response?.status === 401) {
      errorMessage = '登录已过期,请重新登录'
      localStorage.removeItem('token')
      localStorage.removeItem('user')
      window.location.href = '/login'
    } else if (error.response?.data?.message) {
      errorMessage = error.response.data.message
    }
    
    ElMessage.error(errorMessage)
    return Promise.reject(error)
  }
)

export default request
\end{lstlisting}

\section{系统测试}

\subsection{测试策略}
本项目采用多层次的测试策略,确保系统的质量和稳定性:

\begin{itemize}
    \item \textbf{单元测试}:对各个模块和方法进行独立测试
    \item \textbf{集成测试}:测试各模块之间的接口和交互
    \item \textbf{系统测试}:对整个系统进行端到端测试
    \item \textbf{用户验收测试}:验证系统是否满足用户需求
\end{itemize}

\subsection{测试用例}

\subsubsection{功能测试用例}

\textbf{1. 用户登录测试}
\begin{longtable}{|l|p{8cm}|}
\hline
\textbf{测试项目} & 用户登录功能 \\
\hline
\textbf{测试目的} & 验证用户能够正常登录系统 \\
\hline
\textbf{前置条件} & 用户已注册,系统正常运行 \\
\hline
\textbf{测试步骤} & 1. 打开登录页面\\
& 2. 输入正确的用户名和密码\\
& 3. 点击登录按钮 \\
\hline
\textbf{期望结果} & 登录成功,跳转到相应的主页面 \\
\hline
\textbf{测试结果} & ✓ 通过 \\
\hline
\end{longtable}

\textbf{2. 图书搜索测试}
\begin{longtable}{|l|p{8cm}|}
\hline
\textbf{测试项目} & 图书搜索功能 \\
\hline
\textbf{测试目的} & 验证用户能够正常搜索图书 \\
\hline
\textbf{前置条件} & 用户已登录,数据库中有图书数据 \\
\hline
\textbf{测试步骤} & 1. 进入图书浏览页面\\
& 2. 在搜索框中输入关键词\\
& 3. 点击搜索按钮 \\
\hline
\textbf{期望结果} & 显示符合条件的图书列表 \\
\hline
\textbf{测试结果} & ✓ 通过 \\
\hline
\end{longtable}

\textbf{3. 借阅申请测试}
\begin{longtable}{|l|p{8cm}|}
\hline
\textbf{测试项目} & 图书借阅申请功能 \\
\hline
\textbf{测试目的} & 验证读者能够正常提交借阅申请 \\
\hline
\textbf{前置条件} & 读者已登录,选择了可借阅的图书 \\
\hline
\textbf{测试步骤} & 1. 查看图书详情\\
& 2. 点击借阅按钮\\
& 3. 确认借阅申请 \\
\hline
\textbf{期望结果} & 借阅申请提交成功,状态为待审核 \\
\hline
\textbf{测试结果} & ✓ 通过 \\
\hline
\end{longtable}

\subsection{数据库性能测试}

\subsubsection{数据库连接性能测试}
使用MySQL Workbench和自定义测试脚本进行数据库性能测试:

\textbf{1. 基础查询性能测试}
\begin{lstlisting}[language=sql]
-- 单表查询性能测试
SELECT * FROM books WHERE title LIKE '%数据库%';
-- 平均执行时间:15ms (1000条记录)

-- 索引查询性能测试  
SELECT * FROM books WHERE id = 12345;
-- 平均执行时间:2ms (主键索引)

SELECT * FROM users WHERE username = 'admin';
-- 平均执行时间:3ms (唯一索引)
\end{lstlisting}

\textbf{2. 复杂查询性能测试}
\begin{lstlisting}[language=sql]
-- 多表连接查询性能
SELECT u.username, b.title, br.borrow_date, c.name as category
FROM borrow_records br
JOIN users u ON br.user_id = u.id
JOIN books b ON br.book_id = b.id  
JOIN categories c ON b.category_id = c.id
WHERE br.status = 'BORROWED'
ORDER BY br.borrow_date DESC
LIMIT 20;
-- 平均执行时间:45ms (10000条借阅记录)

-- 聚合查询性能
SELECT c.name, COUNT(b.id) as book_count, 
       AVG(b.total_quantity) as avg_quantity
FROM categories c
LEFT JOIN books b ON c.id = b.category_id
GROUP BY c.id, c.name
ORDER BY book_count DESC;
-- 平均执行时间:25ms (100个分类, 5000本书)
\end{lstlisting}

\textbf{3. 数据库并发性能测试}
\begin{itemize}
    \item \textbf{并发读取测试}:50个并发连接同时查询
    \begin{itemize}
        \item 测试场景:同时查询图书列表
        \item 平均响应时间:120ms
        \item 吞吐量:400 查询/秒
        \item CPU使用率:60\%
    \end{itemize}
    
    \item \textbf{并发写入测试}:20个并发连接同时插入
    \begin{itemize}
        \item 测试场景:同时创建借阅记录
        \item 平均响应时间:180ms  
        \item 吞吐量:100 事务/秒
        \item 锁等待时间:平均15ms
    \end{itemize}
    
    \item \textbf{混合负载测试}:70\%读 + 30\%写
    \begin{itemize}
        \item 并发用户:100个
        \item 平均响应时间:150ms
        \item 错误率:0.05\%
        \item 数据库连接池使用率:75\%
    \end{itemize}
\end{itemize}

\subsubsection{事务处理测试}

\textbf{1. ACID特性验证}
\begin{lstlisting}[language=sql]
-- 原子性(Atomicity)测试
START TRANSACTION;
INSERT INTO borrow_records (user_id, book_id, status) VALUES (1, 1, 'PENDING');
UPDATE books SET available_quantity = available_quantity - 1 WHERE id = 1;
-- 模拟异常情况
ROLLBACK; -- 验证所有操作都被回滚

-- 一致性(Consistency)测试  
-- 验证图书可借数量不会变为负数
START TRANSACTION;
SELECT available_quantity FROM books WHERE id = 1 FOR UPDATE;
-- 如果available_quantity <= 0,事务应该失败
COMMIT;

-- 隔离性(Isolation)测试
-- 会话1:读取图书信息
SELECT available_quantity FROM books WHERE id = 1;

-- 会话2:同时更新图书信息
UPDATE books SET available_quantity = available_quantity - 1 WHERE id = 1;

-- 验证隔离级别:READ COMMITTED, REPEATABLE READ

-- 持久性(Durability)测试
-- 提交事务后重启数据库,验证数据持久化
\end{lstlisting}

\textbf{2. 并发控制测试}
\begin{lstlisting}[language=sql]
-- 死锁检测测试
-- 会话1
START TRANSACTION;
UPDATE books SET available_quantity = available_quantity - 1 WHERE id = 1;
UPDATE books SET available_quantity = available_quantity - 1 WHERE id = 2;
COMMIT;

-- 会话2 (同时执行)
START TRANSACTION;  
UPDATE books SET available_quantity = available_quantity - 1 WHERE id = 2;
UPDATE books SET available_quantity = available_quantity - 1 WHERE id = 1;
COMMIT;
-- MySQL会自动检测死锁并回滚其中一个事务
\end{lstlisting}

\textbf{3. 事务性能指标}
\begin{itemize}
    \item \textbf{事务吞吐量}:平均 150 TPS
    \item \textbf{事务响应时间}:平均 80ms (简单事务), 200ms (复杂事务)
    \item \textbf{锁等待时间}:平均 20ms
    \item \textbf{死锁发生率}:0.01\% (高并发场景)
    \item \textbf{事务回滚率}:2\% (包括业务逻辑回滚)
\end{itemize}

\subsubsection{索引效率测试}

\textbf{1. 索引覆盖率分析}
\begin{lstlisting}[language=sql]
-- 分析查询是否使用索引
EXPLAIN SELECT * FROM books WHERE title LIKE '%Spring%';
-- key: idx_books_search, Extra: Using index condition

EXPLAIN SELECT user_id, status FROM borrow_records WHERE status = 'BORROWED';  
-- key: idx_borrow_records_status, Extra: Using index

-- 索引选择性分析
SELECT COUNT(DISTINCT title)/COUNT(*) as title_selectivity,
       COUNT(DISTINCT author)/COUNT(*) as author_selectivity,
       COUNT(DISTINCT isbn)/COUNT(*) as isbn_selectivity
FROM books;
\end{lstlisting}

\textbf{2. 索引性能对比}
\begin{itemize}
    \item \textbf{无索引查询}:全表扫描,5000ms (10万条记录)
    \item \textbf{单列索引}:B+树查询,15ms
    \item \textbf{复合索引}:覆盖索引,8ms
    \item \textbf{唯一索引}:哈希查询,3ms
\end{itemize}

\subsection{应用系统性能测试}

\subsubsection{并发性能测试}
使用JMeter工具对应用系统进行并发性能测试:

\begin{itemize}
    \item \textbf{测试场景}:100个并发用户同时访问系统
    \item \textbf{测试时间}:持续5分钟
    \item \textbf{测试结果}:
    \begin{itemize}
        \item 平均响应时间:1.2秒
        \item 最大响应时间:3.1秒
        \item 错误率:0.1\%
        \item 吞吐量:85 TPS
    \end{itemize}
\end{itemize}

\section{部署与运维}

\subsection{部署环境}

\subsubsection{开发环境}
\begin{itemize}
    \item \textbf{操作系统}:Windows 10/11, macOS, Linux
    \item \textbf{Java版本}:OpenJDK 17+
    \item \textbf{Node.js版本}:16.x+(推荐18.x+)
    \item \textbf{数据库}:MySQL 8.0
    \item \textbf{Maven版本}:3.6+
    \item \textbf{IDE}:IntelliJ IDEA, VS Code
\end{itemize}

\subsubsection{生产环境}
\begin{itemize}
    \item \textbf{服务器}:Linux (CentOS 7/Ubuntu 20.04)
    \item \textbf{Web服务器}:Nginx 1.20+
    \item \textbf{应用服务器}:Spring Boot内嵌Tomcat
    \item \textbf{数据库}:MySQL 8.0 (主从复制)
    \item \textbf{负载均衡}:Nginx + 多实例部署
\end{itemize}

\subsection{部署步骤}

\subsubsection{后端部署}
\begin{enumerate}
    \item 配置生产环境数据库连接
    \item 构建JAR包:\texttt{./mvnw clean package -DskipTests}
    \item 上传JAR包到服务器
    \item 配置systemd服务文件
    \item 启动应用服务
    \item 配置Nginx反向代理
\end{enumerate}

\subsubsection{前端部署}
\begin{enumerate}
    \item 修改生产环境API配置
    \item 构建生产版本:\texttt{npm run build}
    \item 上传dist目录到Web服务器
    \item 配置Nginx静态文件服务
    \item 配置HTTPS证书
\end{enumerate}

\subsection{监控与维护}

\subsubsection{系统监控}
\begin{itemize}
    \item \textbf{应用监控}:Spring Boot Actuator + Prometheus
    \item \textbf{日志监控}:ELK Stack (Elasticsearch + Logstash + Kibana)
    \item \textbf{性能监控}:APM工具监控应用性能
    \item \textbf{基础设施监控}:Zabbix监控服务器资源
\end{itemize}

\subsubsection{备份策略}
\begin{itemize}
    \item \textbf{数据库备份}:每日全量备份 + 实时增量备份
    \item \textbf{应用备份}:定期备份应用配置和代码
    \item \textbf{文件备份}:上传文件的定期备份
    \item \textbf{异地备份}:关键数据的异地存储
\end{itemize}

\section{项目总结}

\subsection{项目成果}

\subsubsection{功能完成情况}
本项目成功实现了预期的所有核心功能:

\begin{itemize}
    \item ✓ 用户认证与权限管理
    \item ✓ 图书信息管理
    \item ✓ 图书分类管理
    \item ✓ 借阅流程管理
    \item ✓ 个人信息管理
    \item ✓ 系统统计功能
    \item ✓ 响应式用户界面
\end{itemize}

\subsubsection{技术亮点}
\begin{enumerate}
    \item \textbf{现代化技术栈}:采用Spring Boot 3.5.0 + Vue 3.5.13等最新技术
    \item \textbf{前后端分离}:清晰的架构设计,便于维护和扩展
    \item \textbf{安全性设计}:JWT 0.12.3认证 + Spring Security权限控制
    \item \textbf{用户体验}:Element Plus 2.8.8组件库 + 响应式设计
    \item \textbf{代码质量}:规范的代码结构和完善的异常处理
    \item \textbf{Composition API}:使用Vue 3 Composition API提升代码复用性
\end{enumerate}

\subsection{数据库系统原理应用总结}

\subsubsection{理论知识的实践应用}
通过本项目的开发,我们深入实践了数据库系统原理课程中的核心理论:

\textbf{1. 数据库设计理论应用}
\begin{itemize}
    \item \textbf{E-R模型设计}:从业务需求出发,识别实体、属性和关系,构建了完整的概念模型
    \item \textbf{关系模式转换}:将E-R图转换为规范的关系模式,正确处理一对多、多对多关系
    \item \textbf{规范化理论}:运用1NF、2NF、3NF规范化理论,消除数据冗余和更新异常
    \item \textbf{反规范化权衡}:在性能和存储之间找到平衡,适度反规范化提升查询效率
\end{itemize}

\textbf{2. 事务管理理论应用}
\begin{itemize}
    \item \textbf{ACID特性实现}:通过存储过程和业务逻辑保证事务的原子性、一致性、隔离性、持久性
    \item \textbf{并发控制机制}:使用行级锁、表级锁解决并发访问冲突,避免脏读、不可重复读、幻读
    \item \textbf{死锁处理}:设计合理的事务顺序,利用MySQL的死锁检测机制自动处理死锁
    \item \textbf{事务隔离级别}:根据业务需求选择合适的隔离级别,平衡一致性和性能
\end{itemize}

\textbf{3. 查询优化理论应用}
\begin{itemize}
    \item \textbf{索引设计原理}:基于查询模式设计B+树索引,提升查询性能
    \item \textbf{查询执行计划}:使用EXPLAIN分析查询执行计划,优化SQL语句
    \item \textbf{统计信息利用}:MySQL优化器基于统计信息选择最优执行策略
    \item \textbf{连接算法优化}:理解嵌套循环连接、哈希连接等算法的适用场景
\end{itemize}

\textbf{4. 存储管理理论应用}
\begin{itemize}
    \item \textbf{存储结构设计}:选择InnoDB存储引擎,利用聚簇索引提升访问效率
    \item \textbf{页面管理}:理解数据页、索引页的组织方式,优化数据布局
    \item \textbf{缓冲区管理}:配置InnoDB缓冲池,提升热数据访问性能
    \item \textbf{日志管理}:利用Redo Log、Undo Log保证事务持久性和恢复能力
\end{itemize}

\subsubsection{数据库系统学习心得}
\begin{itemize}
    \item \textbf{理论与实践结合}:通过实际项目验证了课程理论的正确性和实用性
    \item \textbf{系统性思维}:数据库设计需要统筹考虑功能、性能、安全、可维护性等多个维度
    \item \textbf{性能优化艺术}:索引设计、查询优化需要深入理解数据库内部机制
    \item \textbf{权衡取舍能力}:在规范化与性能、一致性与并发性之间做出合理权衡
    \item \textbf{工程化能力}:掌握了从需求分析到物理实现的完整数据库开发流程
\end{itemize}

\subsection{技术经验总结}

\subsubsection{数据库技术经验}
\begin{itemize}
    \item 掌握了MySQL数据库的高级特性,包括存储引擎、索引、事务、锁机制
    \item 熟练使用数据库设计工具和性能分析工具
    \item 深入理解了ORM框架(JPA/Hibernate)的工作原理和最佳实践
    \item 积累了数据库性能调优和故障排查的实战经验
    \item 学会了数据库安全管理和备份恢复策略
\end{itemize}

\subsubsection{应用开发经验}
\begin{itemize}
    \item 掌握了Spring Boot框架的数据访问层开发技术
    \item 学会了Vue 3 Composition API和现代前端开发模式
    \item 深入理解了JWT认证机制和Spring Security配置
    \item 积累了前后端分离项目的开发和部署经验
    \item 提升了API设计和接口文档编写能力
\end{itemize}

\subsubsection{团队协作经验}
\begin{itemize}
    \item 建立了有效的团队沟通机制
    \item 制定了统一的代码规范和开发流程
    \item 使用Git进行版本控制和协作开发
    \item 实践了敏捷开发的迭代模式
    \item 培养了问题解决和技术分享的能力
\end{itemize}

\subsection{存在的不足}

\subsubsection{功能方面}
\begin{itemize}
    \item 缺少图书推荐算法
    \item 未实现消息通知功能
    \item 报表功能相对简单
    \item 移动端适配有待完善
\end{itemize}

\subsubsection{技术方面}
\begin{itemize}
    \item 缺少缓存机制优化性能
    \item 未实现分布式部署
    \item 日志记录不够完善
    \item 单元测试覆盖率有待提高
\end{itemize}

\subsection{改进方向}

\subsubsection{功能扩展}
\begin{enumerate}
    \item \textbf{智能推荐}:基于用户行为的图书推荐系统
    \item \textbf{移动应用}:开发原生移动应用或小程序
    \item \textbf{社交功能}:图书评论、评分、分享功能
    \item \textbf{数据分析}:更丰富的统计报表和数据可视化
\end{enumerate}

\subsubsection{技术优化}
\begin{enumerate}
    \item \textbf{性能优化}:引入Redis缓存,优化数据库查询
    \item \textbf{微服务化}:将单体应用拆分为微服务架构
    \item \textbf{容器化部署}:使用Docker和Kubernetes部署
    \item \textbf{监控完善}:完善系统监控和告警机制
\end{enumerate}

\section{参考文献}

\begin{enumerate}
    \item 《数据库系统概念》(第7版) - Abraham Silberschatz, Henry F. Korth, S. Sudarshan
    \item 《数据库系统实现》(第2版) - Hector Garcia-Molina, Jeffrey D. Ullman, Jennifer Widom
    \item 《高性能MySQL》(第4版) - Baron Schwartz, Peter Zaitsev, Vadim Tkachenko
    \item 《MySQL技术内幕:InnoDB存储引擎》(第2版) - 姜承尧
    \item 《数据库原理与应用》- 苗雪兰, 刘瑞新, 宋歌
    \item MySQL官方文档. \url{https://dev.mysql.com/doc/}
    \item InnoDB存储引擎架构文档. \url{https://dev.mysql.com/doc/refman/8.0/en/innodb-architecture.html}
    \item Spring Data JPA官方文档. \url{https://spring.io/projects/spring-data-jpa}
    \item 《数据库事务处理的艺术:事务管理与并发控制》- 李海翔
    \item 《SQL优化核心思想》- 罗炳森, 黄超
    \item Spring Boot官方文档. \url{https://spring.io/projects/spring-boot}
    \item Vue.js官方文档. \url{https://vuejs.org/}
    \item Element Plus组件库文档. \url{https://element-plus.org/}
    \item JWT官方规范. \url{https://jwt.io/}
    \item RESTful API设计指南. \url{https://restfulapi.net/}
    \item 《数据库系统原理与设计》- 王珊, 萨师煊
    \item 《数据库查询优化器的艺术:原理解析与SQL性能优化》- 梁敬弘
    \item 数据库规范化理论论文 - E.F. Codd. "A Relational Model of Data for Large Shared Data Banks"
    \item ACID事务属性论文 - Jim Gray, Andreas Reuter. "Transaction Processing: Concepts and Techniques"
    \item B+树索引结构论文 - Rudolf Bayer, Edward M. McCreight. "Organization and Maintenance of Large Ordered Indexes"
\end{enumerate}

\section*{附录}

\subsection*{附录A:系统配置文件}

\subsubsection*{A.1 后端配置文件 (application.properties)}
\begin{lstlisting}[language=properties]
# Spring应用配置
spring.application.name=library

# 数据库配置
spring.datasource.url=jdbc:mysql://localhost:3306/library_db?useUnicode=true&characterEncoding=utf8&useSSL=false&serverTimezone=Asia/Shanghai&allowPublicKeyRetrieval=true&useAffectedRows=true
spring.datasource.username=root
spring.datasource.password=woaimx.0
spring.datasource.driver-class-name=com.mysql.cj.jdbc.Driver

# JPA配置
spring.jpa.hibernate.ddl-auto=validate
spring.jpa.show-sql=false
spring.jpa.properties.hibernate.dialect=org.hibernate.dialect.MySQL8Dialect
spring.jpa.properties.hibernate.format_sql=true
spring.jpa.database-platform=org.hibernate.dialect.MySQL8Dialect

# JWT配置
jwt.secret=mySecretKey123456789012345678901234567890
jwt.expiration=86400000

# 服务器配置
server.port=8080

# CORS配置 - 允许本地和局域网访问
cors.allowed-origins=http://localhost:5173,http://localhost:5174,http://localhost:5175,http://localhost:5176,http://localhost:5177,http://localhost:5178,http://localhost:5179,http://localhost:5180,http://192.168.0.100:5173,http://192.168.0.100:5174,http://192.168.0.100:5175,http://192.168.0.100:5176,http://192.168.0.100:5177,http://192.168.0.100:5178,http://192.168.0.100:5179,http://192.168.0.100:5180
\end{lstlisting}

\subsubsection*{A.2 前端配置文件 (package.json)}
\begin{lstlisting}[language=json]
{
  "name": "library-ui",
  "private": true,
  "version": "0.0.0",
  "type": "module",
  "scripts": {
    "dev": "vite --host",
    "build": "vite build",
    "preview": "vite preview"
  },
  "dependencies": {
    "@element-plus/icons-vue": "^2.3.1",
    "@vueuse/core": "^11.3.0",
    "axios": "^1.7.9",
    "dayjs": "^1.11.13",
    "element-plus": "^2.8.8",
    "pinia": "^2.3.0",
    "vue": "^3.5.13",
    "vue-router": "^4.5.0"
  },
  "devDependencies": {
    "@vitejs/plugin-vue": "^5.2.3",
    "vite": "^6.3.5"
  }
}
\end{lstlisting}

\subsubsection*{A.3 Maven配置文件 (pom.xml)}
\begin{lstlisting}[language=xml]
<?xml version="1.0" encoding="UTF-8"?>
<project xmlns="http://maven.apache.org/POM/4.0.0" 
         xmlns:xsi="http://www.w3.org/2001/XMLSchema-instance"
         xsi:schemaLocation="http://maven.apache.org/POM/4.0.0 
         https://maven.apache.org/xsd/maven-4.0.0.xsd">
    <modelVersion>4.0.0</modelVersion>
    <parent>
        <groupId>org.springframework.boot</groupId>
        <artifactId>spring-boot-starter-parent</artifactId>
        <version>3.5.0</version>
        <relativePath/>
    </parent>
    <groupId>com.example</groupId>
    <artifactId>library</artifactId>
    <version>0.0.1-SNAPSHOT</version>
    <name>library</name>
    <description>library</description>
    
    <properties>
        <java.version>17</java.version>
    </properties>
    
    <dependencies>
        <!-- Spring Boot Web -->
        <dependency>
            <groupId>org.springframework.boot</groupId>
            <artifactId>spring-boot-starter-web</artifactId>
        </dependency>
        
        <!-- Spring Boot JPA -->
        <dependency>
            <groupId>org.springframework.boot</groupId>
            <artifactId>spring-boot-starter-data-jpa</artifactId>
        </dependency>
        
        <!-- MySQL Driver -->
        <dependency>
            <groupId>com.mysql</groupId>
            <artifactId>mysql-connector-j</artifactId>
            <scope>runtime</scope>
        </dependency>
        
        <!-- Spring Boot Security -->
        <dependency>
            <groupId>org.springframework.boot</groupId>
            <artifactId>spring-boot-starter-security</artifactId>
        </dependency>
        
        <!-- JWT -->
        <dependency>
            <groupId>io.jsonwebtoken</groupId>
            <artifactId>jjwt-api</artifactId>
            <version>0.12.3</version>
        </dependency>
        <dependency>
            <groupId>io.jsonwebtoken</groupId>
            <artifactId>jjwt-impl</artifactId>
            <version>0.12.3</version>
            <scope>runtime</scope>
        </dependency>
        <dependency>
            <groupId>io.jsonwebtoken</groupId>
            <artifactId>jjwt-jackson</artifactId>
            <version>0.12.3</version>
            <scope>runtime</scope>
        </dependency>
        
        <!-- Validation -->
        <dependency>
            <groupId>org.springframework.boot</groupId>
            <artifactId>spring-boot-starter-validation</artifactId>
        </dependency>
        
        <!-- Lombok -->
        <dependency>
            <groupId>org.projectlombok</groupId>
            <artifactId>lombok</artifactId>
            <optional>true</optional>
        </dependency>
        
        <!-- Test Dependencies -->
        <dependency>
            <groupId>org.springframework.boot</groupId>
            <artifactId>spring-boot-starter-test</artifactId>
            <scope>test</scope>
        </dependency>
        <dependency>
            <groupId>org.springframework.security</groupId>
            <artifactId>spring-security-test</artifactId>
            <scope>test</scope>
        </dependency>
    </dependencies>
    
    <build>
        <plugins>
            <plugin>
                <groupId>org.springframework.boot</groupId>
                <artifactId>spring-boot-maven-plugin</artifactId>
                <configuration>
                    <excludes>
                        <exclude>
                            <groupId>org.projectlombok</groupId>
                            <artifactId>lombok</artifactId>
                        </exclude>
                    </excludes>
                </configuration>
            </plugin>
        </plugins>
    </build>
</project>
\end{lstlisting}

\subsection*{附录B:数据库初始化脚本}

\subsubsection*{B.1 数据库和表结构创建}
\begin{lstlisting}[language=sql]
-- 创建数据库
CREATE DATABASE library_db CHARACTER SET utf8mb4 COLLATE utf8mb4_unicode_ci;
USE library_db;

-- 创建用户表
CREATE TABLE users (
    id BIGINT AUTO_INCREMENT PRIMARY KEY,
    username VARCHAR(20) NOT NULL UNIQUE,
    password VARCHAR(255) NOT NULL,
    real_name VARCHAR(50) NOT NULL,
    email VARCHAR(100) UNIQUE,
    phone VARCHAR(20),
    role ENUM('ADMIN', 'READER') NOT NULL,
    enabled BOOLEAN DEFAULT TRUE,
    created_at TIMESTAMP DEFAULT CURRENT_TIMESTAMP,
    updated_at TIMESTAMP DEFAULT CURRENT_TIMESTAMP ON UPDATE CURRENT_TIMESTAMP,
    INDEX idx_users_username (username),
    INDEX idx_users_email (email),
    INDEX idx_users_role (role)
) ENGINE=InnoDB;

-- 创建分类表
CREATE TABLE categories (
    id BIGINT AUTO_INCREMENT PRIMARY KEY,
    name VARCHAR(50) NOT NULL UNIQUE,
    description VARCHAR(200),
    created_at TIMESTAMP DEFAULT CURRENT_TIMESTAMP,
    updated_at TIMESTAMP DEFAULT CURRENT_TIMESTAMP ON UPDATE CURRENT_TIMESTAMP,
    INDEX idx_categories_name (name)
) ENGINE=InnoDB;

-- 创建图书表
CREATE TABLE books (
    id BIGINT AUTO_INCREMENT PRIMARY KEY,
    title VARCHAR(200) NOT NULL,
    author VARCHAR(100) NOT NULL,
    isbn VARCHAR(20) UNIQUE,
    publisher VARCHAR(100),
    publish_date VARCHAR(20),
    description TEXT,
    cover_image VARCHAR(500),
    total_quantity INT NOT NULL DEFAULT 0,
    available_quantity INT NOT NULL DEFAULT 0,
    price DECIMAL(10,2),
    category_id BIGINT,
    status ENUM('AVAILABLE', 'UNAVAILABLE', 'MAINTENANCE') DEFAULT 'AVAILABLE',
    created_at TIMESTAMP DEFAULT CURRENT_TIMESTAMP,
    updated_at TIMESTAMP DEFAULT CURRENT_TIMESTAMP ON UPDATE CURRENT_TIMESTAMP,
    FOREIGN KEY (category_id) REFERENCES categories(id) ON DELETE SET NULL,
    INDEX idx_books_title (title),
    INDEX idx_books_author (author),
    INDEX idx_books_isbn (isbn),
    INDEX idx_books_category (category_id),
    INDEX idx_books_search (title, author, publisher),
    CONSTRAINT chk_books_quantity CHECK (available_quantity >= 0 AND available_quantity <= total_quantity)
) ENGINE=InnoDB;

-- 创建借阅记录表
CREATE TABLE borrow_records (
    id BIGINT AUTO_INCREMENT PRIMARY KEY,
    user_id BIGINT,
    book_id BIGINT NOT NULL,
    status ENUM('PENDING', 'APPROVED', 'BORROWED', 'RETURNED', 'OVERDUE', 'REJECTED') NOT NULL,
    borrow_date TIMESTAMP,
    due_date TIMESTAMP,
    return_date TIMESTAMP,
    remarks VARCHAR(500),
    created_at TIMESTAMP DEFAULT CURRENT_TIMESTAMP,
    updated_at TIMESTAMP DEFAULT CURRENT_TIMESTAMP ON UPDATE CURRENT_TIMESTAMP,
    FOREIGN KEY (user_id) REFERENCES users(id) ON DELETE SET NULL,
    FOREIGN KEY (book_id) REFERENCES books(id) ON DELETE RESTRICT,
    INDEX idx_borrow_user (user_id),
    INDEX idx_borrow_book (book_id),
    INDEX idx_borrow_status (status),
    INDEX idx_borrow_dates (borrow_date, due_date),
    INDEX idx_borrow_user_status (user_id, status, created_at),
    CONSTRAINT chk_borrow_dates CHECK (due_date > borrow_date AND (return_date IS NULL OR return_date >= borrow_date))
) ENGINE=InnoDB;
\end{lstlisting}

\subsubsection*{B.2 初始数据插入}
\begin{lstlisting}[language=sql]
-- 插入管理员用户
INSERT INTO users (username, password, real_name, email, role) VALUES
('admin', '$2a$10$N.zmdr9k7uOCQb376NoUnuTJ8iAt6Z5EHsM8lE9lBOsl7iYrCzk3C', 
 '系统管理员', 'admin@library.com', 'ADMIN'),
('reader1', '$2a$10$N.zmdr9k7uOCQb376NoUnuTJ8iAt6Z5EHsM8lE9lBOsl7iYrCzk3C', 
 '张三', 'zhangsan@email.com', 'READER'),
('reader2', '$2a$10$N.zmdr9k7uOCQb376NoUnuTJ8iAt6Z5EHsM8lE9lBOsl7iYrCzk3C', 
 '李四', 'lisi@email.com', 'READER');

-- 插入图书分类
INSERT INTO categories (name, description) VALUES
('计算机科学', '计算机相关的技术书籍'),
('数据库技术', '数据库系统、SQL、NoSQL等相关书籍'),
('编程语言', 'Java、Python、C++等编程语言书籍'),
('人工智能', '机器学习、深度学习、AI算法等'),
('软件工程', '软件开发方法论、项目管理等'),
('网络技术', '计算机网络、网络安全等'),
('操作系统', 'Linux、Windows、操作系统原理等'),
('文学小说', '文学作品、小说等'),
('历史传记', '历史、传记类书籍'),
('科学技术', '自然科学、工程技术等');

-- 插入图书数据
INSERT INTO books (title, author, isbn, publisher, publish_date, description, total_quantity, available_quantity, category_id) VALUES
('数据库系统概念', 'Abraham Silberschatz', '9787111544889', '机械工业出版社', '2019-01', '数据库系统的经典教材,全面介绍数据库理论与实践', 5, 5, 2),
('MySQL技术内幕', '姜承尧', '9787121198126', '电子工业出版社', '2020-03', '深入解析MySQL InnoDB存储引擎', 3, 3, 2),
('Java编程思想', 'Bruce Eckel', '9787111213826', '机械工业出版社', '2018-06', 'Java编程的经典教材', 8, 7, 3),
('算法导论', 'Thomas H. Cormen', '9787111407010', '机械工业出版社', '2019-12', '计算机算法的权威教材', 4, 4, 1),
('深度学习', 'Ian Goodfellow', '9787115461476', '人民邮电出版社', '2020-05', '深度学习领域的重要著作', 6, 5, 4),
('Spring Boot实战', 'Craig Walls', '9787115422835', '人民邮电出版社', '2021-08', 'Spring Boot开发实践指南', 10, 9, 3),
('计算机网络', '谢希仁', '9787121302954', '电子工业出版社', '2019-09', '计算机网络原理教材', 7, 7, 6),
('操作系统概念', 'Abraham Silberschatz', '9787111544876', '机械工业出版社', '2020-01', '操作系统理论与实践', 5, 4, 7),
('软件工程', 'Ian Sommerville', '9787111555568', '机械工业出版社', '2021-03', '软件工程方法论', 3, 3, 5),
('Python编程从入门到实践', 'Eric Matthes', '9787115428028', '人民邮电出版社', '2020-11', 'Python编程入门教程', 12, 11, 3);
\end{lstlisting}

\subsubsection*{B.3 存储过程和触发器}
\begin{lstlisting}[language=sql]
-- 图书借阅存储过程
DELIMITER //
CREATE PROCEDURE sp_BorrowBook(
    IN p_user_id BIGINT,
    IN p_book_id BIGINT,
    OUT p_result VARCHAR(100)
)
BEGIN
    DECLARE book_available INT DEFAULT 0;
    DECLARE user_borrow_count INT DEFAULT 0;
    DECLARE EXIT HANDLER FOR SQLEXCEPTION
    BEGIN
        ROLLBACK;
        SET p_result = 'Error: Transaction failed';
    END;

    START TRANSACTION;
    
    SELECT available_quantity INTO book_available 
    FROM books WHERE id = p_book_id FOR UPDATE;
    
    SELECT COUNT(*) INTO user_borrow_count 
    FROM borrow_records 
    WHERE user_id = p_user_id AND status IN ('PENDING', 'APPROVED', 'BORROWED');
    
    IF book_available <= 0 THEN
        SET p_result = 'Error: Book not available';
        ROLLBACK;
    ELSEIF user_borrow_count >= 5 THEN
        SET p_result = 'Error: User borrow limit exceeded';
        ROLLBACK;
    ELSE
        INSERT INTO borrow_records (user_id, book_id, status, created_at)
        VALUES (p_user_id, p_book_id, 'PENDING', NOW());
        
        SET p_result = 'Success: Borrow request created';
        COMMIT;
    END IF;
END //
DELIMITER ;

-- 图书数量更新触发器
DELIMITER //
CREATE TRIGGER tr_UpdateBookQuantity
AFTER UPDATE ON borrow_records
FOR EACH ROW
BEGIN
    IF OLD.status IN ('PENDING', 'APPROVED') AND NEW.status = 'BORROWED' THEN
        UPDATE books 
        SET available_quantity = available_quantity - 1 
        WHERE id = NEW.book_id;
    END IF;
    
    IF OLD.status = 'BORROWED' AND NEW.status = 'RETURNED' THEN
        UPDATE books 
        SET available_quantity = available_quantity + 1 
        WHERE id = NEW.book_id;
    END IF;
END //
DELIMITER ;

-- 启用事件调度器
SET GLOBAL event_scheduler = ON;

-- 逾期检查事件
DELIMITER //
CREATE EVENT ev_CheckOverdueBooks
ON SCHEDULE EVERY 1 DAY
STARTS CURRENT_TIMESTAMP
DO
BEGIN
    UPDATE borrow_records 
    SET status = 'OVERDUE'
    WHERE status = 'BORROWED' 
    AND due_date < CURDATE()
    AND return_date IS NULL;
END //
DELIMITER ;
\end{lstlisting}

\subsection*{附录C:API接口文档}

详细的API接口文档已整合在系统设计章节中,包含了所有主要的RESTful API端点、请求参数和响应格式。

\end{document} 