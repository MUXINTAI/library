\documentclass[12pt,a4paper]{article}
\usepackage{xeCJK}
\usepackage{geometry}
\usepackage{graphicx}
\usepackage{hyperref}
\usepackage{listings}
\usepackage{xcolor}
\usepackage{booktabs}
\usepackage{array}
\usepackage{multirow}
\usepackage{longtable}
\usepackage{fancyhdr}
\usepackage{titlesec}
\usepackage{enumitem}
\usepackage{float}
\usepackage{amsmath}
\usepackage{amsfonts}
\usepackage{amssymb}

% 页面设置
\geometry{left=2.5cm,right=2.5cm,top=2.5cm,bottom=2.5cm}

% 中文字体设置
\setCJKmainfont{SimSun}
\setCJKsansfont{SimHei}
\setCJKmonofont{FangSong}

% 页眉页脚设置
\pagestyle{fancy}
\fancyhf{}
\fancyhead[C]{图书管理系统项目报告}
\fancyfoot[C]{\thepage}

% 标题格式设置
\titleformat{\section}{\Large\bfseries}{\thesection}{1em}{}
\titleformat{\subsection}{\large\bfseries}{\thesubsection}{1em}{}
\titleformat{\subsubsection}{\normalsize\bfseries}{\thesubsubsection}{1em}{}

% 代码块设置
\lstset{
    basicstyle=\ttfamily\small,
    keywordstyle=\color{blue}\bfseries,
    commentstyle=\color{green!60!black},
    stringstyle=\color{red},
    showstringspaces=false,
    breaklines=true,
    frame=single,
    backgroundcolor=\color{gray!10},
    numbers=left,
    numberstyle=\tiny\color{gray},
    stepnumber=1,
    numbersep=10pt,
    tabsize=4
}

% 超链接设置
\hypersetup{
    colorlinks=true,
    linkcolor=black,
    urlcolor=blue,
    citecolor=blue
}

\title{\textbf{\Huge 图书管理系统\\项目开发报告}}
\author{}
\date{}

\begin{document}

\maketitle
\thispagestyle{empty}

\vspace{2cm}

\begin{center}
\Large
\textbf{基于Spring Boot + Vue 3的现代化图书管理系统}

\vspace{1cm}

\normalsize
课程名称:软件工程实践\\
指导教师:XXX教授\\
完成时间:\today

\vspace{3cm}

\begin{table}[H]
\centering
\begin{tabular}{|c|c|c|c|}
\hline
\textbf{角色} & \textbf{姓名} & \textbf{学号} & \textbf{主要分工} \\
\hline
组长 & 冯达 & 072108208 & 项目架构设计、后端核心开发、系统集成 \\
\hline
组员A & 张三 & 123123 & 前端界面开发、用户体验设计、前后端联调 \\
\hline
组员B & 李四 & 4564546 & 数据库设计、API接口开发、系统测试 \\
\hline
组员C & 王五 & 789789 & 需求分析、文档编写、部署运维 \\
\hline
\end{tabular}
\end{table}

\end{center}

\newpage
\tableofcontents
\newpage

\section{项目概述}

\subsection{项目背景}
随着信息技术的快速发展,传统的图书管理方式已经无法满足现代图书馆的管理需求。为了提高图书管理效率,减少人工操作成本,提升用户体验,我们开发了这套基于Spring Boot和Vue 3的现代化图书管理系统。

\subsection{项目目标}
\begin{itemize}
    \item 实现图书信息的数字化管理
    \item 提供便捷的图书借阅和归还服务
    \item 支持多角色权限管理
    \item 提供友好的用户界面和良好的用户体验
    \item 确保系统的安全性和稳定性
\end{itemize}

\subsection{项目特色}
\begin{itemize}
    \item \textbf{前后端分离架构}:采用现代化的前后端分离设计,提高系统可维护性
    \item \textbf{响应式设计}:支持多种设备访问,适配PC端和移动端
    \item \textbf{角色权限控制}:支持管理员和读者两种角色,权限分离明确
    \item \textbf{实时数据更新}:基于RESTful API,实现前后端数据实时同步
    \item \textbf{安全认证机制}:采用JWT令牌认证,确保系统安全
\end{itemize}

\section{需求分析}

\subsection{功能性需求}

\subsubsection{管理员功能需求}
\begin{enumerate}
    \item \textbf{系统管理}
    \begin{itemize}
        \item 用户账户管理(增删改查)
        \item 权限分配和管理
        \item 系统参数配置
    \end{itemize}
    
    \item \textbf{图书管理}
    \begin{itemize}
        \item 图书信息录入、修改、删除
        \item 图书分类管理
        \item 图书库存管理
        \item 图书搜索和查询
    \end{itemize}
    
    \item \textbf{借阅管理}
    \begin{itemize}
        \item 借阅申请审核
        \item 图书归还处理
        \item 逾期图书管理
        \item 借阅统计和报表
    \end{itemize}
    
    \item \textbf{数据统计}
    \begin{itemize}
        \item 图书借阅统计
        \item 用户活跃度统计
        \item 系统使用情况分析
    \end{itemize}
\end{enumerate}

\subsubsection{读者功能需求}
\begin{enumerate}
    \item \textbf{账户管理}
    \begin{itemize}
        \item 用户注册和登录
        \item 个人信息维护
        \item 密码修改
    \end{itemize}
    
    \item \textbf{图书浏览}
    \begin{itemize}
        \item 图书搜索和筛选
        \item 图书详情查看
        \item 图书分类浏览
    \end{itemize}
    
    \item \textbf{借阅服务}
    \begin{itemize}
        \item 在线借阅申请
        \item 借阅记录查询
        \item 借阅申请取消
        \item 借阅到期提醒
    \end{itemize}
\end{enumerate}

\subsection{非功能性需求}

\subsubsection{性能需求}
\begin{itemize}
    \item 系统响应时间不超过3秒
    \item 支持并发用户数不少于100人
    \item 数据库查询效率优化
    \item 前端页面加载速度优化
\end{itemize}

\subsubsection{安全需求}
\begin{itemize}
    \item 用户身份认证和授权
    \item 数据传输加密
    \item SQL注入防护
    \item XSS攻击防护
    \item 敏感数据保护
\end{itemize}

\subsubsection{可用性需求}
\begin{itemize}
    \item 系统可用性达到99\%以上
    \item 支持7×24小时运行
    \item 具备故障恢复能力
    \item 支持数据备份和恢复
\end{itemize}

\section{系统设计}

\subsection{系统架构}

本系统采用前后端分离的架构设计,主要包含以下几个层次:

\begin{figure}[H]
\centering
\begin{tabular}{|c|}
\hline
\textbf{前端展示层 (Vue 3 + Element Plus)} \\
\hline
用户界面 | 交互逻辑 | 状态管理 \\
\hline
\hline
\textbf{API接口层 (RESTful API)} \\
\hline
HTTP请求/响应 | JSON数据格式 | 状态码 \\
\hline
\hline
\textbf{业务逻辑层 (Spring Boot)} \\
\hline
控制器 | 服务层 | 业务逻辑处理 \\
\hline
\hline
\textbf{数据访问层 (Spring Data JPA)} \\
\hline
实体映射 | 数据库操作 | 事务管理 \\
\hline
\hline
\textbf{数据存储层 (MySQL)} \\
\hline
数据持久化 | 数据完整性 | 索引优化 \\
\hline
\end{tabular}
\caption{系统架构图}
\end{figure}

\subsection{技术栈}

\subsubsection{后端技术栈}
\begin{itemize}
    \item \textbf{Spring Boot 3.x}:主框架,提供依赖注入、自动配置等功能
    \item \textbf{Spring Security}:安全框架,处理认证和授权
    \item \textbf{Spring Data JPA}:数据访问层,简化数据库操作
    \item \textbf{MySQL 8.0}:关系型数据库,存储业务数据
    \item \textbf{JWT}:无状态身份验证令牌
    \item \textbf{Maven}:项目构建和依赖管理工具
\end{itemize}

\subsubsection{前端技术栈}
\begin{itemize}
    \item \textbf{Vue 3}:渐进式JavaScript框架
    \item \textbf{Element Plus}:基于Vue 3的UI组件库
    \item \textbf{Pinia}:Vue 3的状态管理库
    \item \textbf{Vue Router}:Vue.js官方路由管理器
    \item \textbf{Axios}:基于Promise的HTTP客户端
    \item \textbf{Vite}:快速的前端构建工具
\end{itemize}

\subsection{数据库设计}

\subsubsection{数据库概念模型}
系统主要包含以下实体及其关系:
\begin{itemize}
    \item \textbf{用户(User)}:系统用户信息
    \item \textbf{图书(Book)}:图书基本信息
    \item \textbf{分类(Category)}:图书分类信息
    \item \textbf{借阅记录(BorrowRecord)}:借阅相关信息
\end{itemize}

\subsubsection{数据表设计}

\textbf{1. 用户表(users)}
\begin{longtable}{|l|l|l|l|}
\hline
\textbf{字段名} & \textbf{类型} & \textbf{约束} & \textbf{说明} \\
\hline
id & BIGINT & PRIMARY KEY, AUTO\_INCREMENT & 用户ID \\
\hline
username & VARCHAR(50) & UNIQUE, NOT NULL & 用户名 \\
\hline
password & VARCHAR(255) & NOT NULL & 密码(加密) \\
\hline
real\_name & VARCHAR(100) & NOT NULL & 真实姓名 \\
\hline
email & VARCHAR(100) & UNIQUE & 邮箱地址 \\
\hline
phone & VARCHAR(20) & & 电话号码 \\
\hline
role & ENUM('ADMIN','READER') & NOT NULL & 用户角色 \\
\hline
enabled & BOOLEAN & DEFAULT TRUE & 是否启用 \\
\hline
created\_at & TIMESTAMP & DEFAULT CURRENT\_TIMESTAMP & 创建时间 \\
\hline
updated\_at & TIMESTAMP & DEFAULT CURRENT\_TIMESTAMP ON UPDATE CURRENT\_TIMESTAMP & 更新时间 \\
\hline
\end{longtable}

\textbf{2. 图书表(books)}
\begin{longtable}{|l|l|l|l|}
\hline
\textbf{字段名} & \textbf{类型} & \textbf{约束} & \textbf{说明} \\
\hline
id & BIGINT & PRIMARY KEY, AUTO\_INCREMENT & 图书ID \\
\hline
title & VARCHAR(200) & NOT NULL & 书名 \\
\hline
author & VARCHAR(100) & NOT NULL & 作者 \\
\hline
isbn & VARCHAR(20) & UNIQUE & ISBN号 \\
\hline
publisher & VARCHAR(100) & & 出版社 \\
\hline
publish\_date & DATE & & 出版日期 \\
\hline
category\_id & BIGINT & FOREIGN KEY & 分类ID \\
\hline
total\_quantity & INT & NOT NULL, DEFAULT 1 & 总数量 \\
\hline
available\_quantity & INT & NOT NULL, DEFAULT 1 & 可借数量 \\
\hline
description & TEXT & & 图书描述 \\
\hline
created\_at & TIMESTAMP & DEFAULT CURRENT\_TIMESTAMP & 创建时间 \\
\hline
updated\_at & TIMESTAMP & DEFAULT CURRENT\_TIMESTAMP ON UPDATE CURRENT\_TIMESTAMP & 更新时间 \\
\hline
\end{longtable}

\textbf{3. 分类表(categories)}
\begin{longtable}{|l|l|l|l|}
\hline
\textbf{字段名} & \textbf{类型} & \textbf{约束} & \textbf{说明} \\
\hline
id & BIGINT & PRIMARY KEY, AUTO\_INCREMENT & 分类ID \\
\hline
name & VARCHAR(50) & UNIQUE, NOT NULL & 分类名称 \\
\hline
description & TEXT & & 分类描述 \\
\hline
created\_at & TIMESTAMP & DEFAULT CURRENT\_TIMESTAMP & 创建时间 \\
\hline
updated\_at & TIMESTAMP & DEFAULT CURRENT\_TIMESTAMP ON UPDATE CURRENT\_TIMESTAMP & 更新时间 \\
\hline
\end{longtable}

\textbf{4. 借阅记录表(borrow\_records)}
\begin{longtable}{|l|l|l|l|}
\hline
\textbf{字段名} & \textbf{类型} & \textbf{约束} & \textbf{说明} \\
\hline
id & BIGINT & PRIMARY KEY, AUTO\_INCREMENT & 记录ID \\
\hline
user\_id & BIGINT & FOREIGN KEY, NOT NULL & 用户ID \\
\hline
book\_id & BIGINT & FOREIGN KEY, NOT NULL & 图书ID \\
\hline
status & ENUM('PENDING','BORROWED','RETURNED','OVERDUE') & NOT NULL & 借阅状态 \\
\hline
borrow\_date & DATE & & 借阅日期 \\
\hline
due\_date & DATE & & 应还日期 \\
\hline
return\_date & DATE & & 实际归还日期 \\
\hline
remarks & TEXT & & 备注信息 \\
\hline
created\_at & TIMESTAMP & DEFAULT CURRENT\_TIMESTAMP & 创建时间 \\
\hline
updated\_at & TIMESTAMP & DEFAULT CURRENT\_TIMESTAMP ON UPDATE CURRENT\_TIMESTAMP & 更新时间 \\
\hline
\end{longtable}

\section{系统实现}

\subsection{后端实现}

\subsubsection{项目结构}
\begin{lstlisting}[language=bash]
src/main/java/com/example/library/
├── config/                     # 配置类
│   ├── SecurityConfig.java    # 安全配置
│   ├── GlobalExceptionHandler.java # 全局异常处理
│   └── DataInitializer.java   # 数据初始化
├── controller/                 # 控制器层
│   ├── AuthController.java    # 认证控制器
│   ├── AdminController.java   # 管理员控制器
│   ├── ReaderController.java  # 读者控制器
│   └── BookController.java    # 图书控制器
├── dto/                       # 数据传输对象
├── entity/                    # 实体类
├── repository/                # 数据访问层
├── service/                   # 服务层
├── filter/                    # 过滤器
└── util/                      # 工具类
\end{lstlisting}

\subsubsection{核心实现}

\textbf{1. 安全配置}
\begin{lstlisting}[language=java]
@Configuration
@EnableWebSecurity
public class SecurityConfig {
    
    @Bean
    public SecurityFilterChain filterChain(HttpSecurity http) throws Exception {
        http.csrf(csrf -> csrf.disable())
            .cors(cors -> cors.configurationSource(corsConfigurationSource()))
            .sessionManagement(session -> 
                session.sessionCreationPolicy(SessionCreationPolicy.STATELESS))
            .authorizeHttpRequests(auth -> auth
                .requestMatchers("/api/auth/**").permitAll()
                .requestMatchers("/api/admin/**").hasRole("ADMIN")
                .requestMatchers("/api/reader/**").hasAnyRole("READER", "ADMIN")
                .anyRequest().authenticated()
            )
            .addFilterBefore(jwtAuthenticationFilter, 
                UsernamePasswordAuthenticationFilter.class);
        
        return http.build();
    }
}
\end{lstlisting}

\textbf{2. JWT工具类}
\begin{lstlisting}[language=java]
@Component
public class JwtUtil {
    
    @Value("${jwt.secret}")
    private String secret;
    
    @Value("${jwt.expiration}")
    private Long expiration;
    
    public String generateToken(String username, String role) {
        return Jwts.builder()
                .setSubject(username)
                .claim("role", role)
                .setIssuedAt(new Date())
                .setExpiration(new Date(System.currentTimeMillis() + expiration))
                .signWith(SignatureAlgorithm.HS256, secret)
                .compact();
    }
    
    public boolean validateToken(String token) {
        try {
            Jwts.parser().setSigningKey(secret).parseClaimsJws(token);
            return true;
        } catch (JwtException | IllegalArgumentException e) {
            return false;
        }
    }
}
\end{lstlisting}

\subsection{前端实现}

\subsubsection{项目结构}
\begin{lstlisting}[language=bash]
library-ui/src/
├── components/          # 公共组件
├── views/              # 页面组件
│   ├── admin/          # 管理员页面
│   │   ├── BookManagement.vue
│   │   ├── CategoryManagement.vue
│   │   ├── BorrowManagement.vue
│   │   └── UserManagement.vue
│   ├── reader/         # 读者页面
│   │   ├── BookList.vue
│   │   ├── MyBorrows.vue
│   │   └── Profile.vue
│   ├── Dashboard.vue   # 仪表盘
│   └── Login.vue       # 登录页面
├── router/             # 路由配置
├── stores/             # 状态管理
├── utils/              # 工具函数
└── style.css          # 全局样式
\end{lstlisting}

\subsubsection{核心实现}

\textbf{1. 状态管理}
\begin{lstlisting}[language=javascript]
import { defineStore } from 'pinia'

export const useAuthStore = defineStore('auth', {
  state: () => ({
    token: localStorage.getItem('token'),
    user: JSON.parse(localStorage.getItem('user') || 'null')
  }),
  
  getters: {
    isAuthenticated: (state) => !!state.token,
    isAdmin: (state) => state.user?.role === 'ADMIN',
    isReader: (state) => state.user?.role === 'READER'
  },
  
  actions: {
    login(token, user) {
      this.token = token
      this.user = user
      localStorage.setItem('token', token)
      localStorage.setItem('user', JSON.stringify(user))
    },
    
    logout() {
      this.token = null
      this.user = null
      localStorage.removeItem('token')
      localStorage.removeItem('user')
    }
  }
})
\end{lstlisting}

\textbf{2. API请求封装}
\begin{lstlisting}[language=javascript]
import axios from 'axios'
import { useAuthStore } from '@/stores/auth'

const api = axios.create({
  baseURL: 'http://localhost:8080/api',
  timeout: 10000
})

// 请求拦截器
api.interceptors.request.use(
  config => {
    const authStore = useAuthStore()
    if (authStore.token) {
      config.headers.Authorization = `Bearer ${authStore.token}`
    }
    return config
  },
  error => Promise.reject(error)
)

// 响应拦截器
api.interceptors.response.use(
  response => response.data,
  error => {
    if (error.response?.status === 401) {
      const authStore = useAuthStore()
      authStore.logout()
      router.push('/login')
    }
    return Promise.reject(error)
  }
)

export default api
\end{lstlisting}

\section{系统测试}

\subsection{测试策略}
本项目采用多层次的测试策略,确保系统的质量和稳定性:

\begin{itemize}
    \item \textbf{单元测试}:对各个模块和方法进行独立测试
    \item \textbf{集成测试}:测试各模块之间的接口和交互
    \item \textbf{系统测试}:对整个系统进行端到端测试
    \item \textbf{用户验收测试}:验证系统是否满足用户需求
\end{itemize}

\subsection{测试用例}

\subsubsection{功能测试用例}

\textbf{1. 用户登录测试}
\begin{longtable}{|l|p{8cm}|}
\hline
\textbf{测试项目} & 用户登录功能 \\
\hline
\textbf{测试目的} & 验证用户能够正常登录系统 \\
\hline
\textbf{前置条件} & 用户已注册,系统正常运行 \\
\hline
\textbf{测试步骤} & 1. 打开登录页面\\
& 2. 输入正确的用户名和密码\\
& 3. 点击登录按钮 \\
\hline
\textbf{期望结果} & 登录成功,跳转到相应的主页面 \\
\hline
\textbf{测试结果} & ✓ 通过 \\
\hline
\end{longtable}

\textbf{2. 图书搜索测试}
\begin{longtable}{|l|p{8cm}|}
\hline
\textbf{测试项目} & 图书搜索功能 \\
\hline
\textbf{测试目的} & 验证用户能够正常搜索图书 \\
\hline
\textbf{前置条件} & 用户已登录,数据库中有图书数据 \\
\hline
\textbf{测试步骤} & 1. 进入图书浏览页面\\
& 2. 在搜索框中输入关键词\\
& 3. 点击搜索按钮 \\
\hline
\textbf{期望结果} & 显示符合条件的图书列表 \\
\hline
\textbf{测试结果} & ✓ 通过 \\
\hline
\end{longtable}

\textbf{3. 借阅申请测试}
\begin{longtable}{|l|p{8cm}|}
\hline
\textbf{测试项目} & 图书借阅申请功能 \\
\hline
\textbf{测试目的} & 验证读者能够正常提交借阅申请 \\
\hline
\textbf{前置条件} & 读者已登录,选择了可借阅的图书 \\
\hline
\textbf{测试步骤} & 1. 查看图书详情\\
& 2. 点击借阅按钮\\
& 3. 确认借阅申请 \\
\hline
\textbf{期望结果} & 借阅申请提交成功,状态为待审核 \\
\hline
\textbf{测试结果} & ✓ 通过 \\
\hline
\end{longtable}

\subsection{性能测试}

\subsubsection{并发性能测试}
使用JMeter工具对系统进行并发性能测试:

\begin{itemize}
    \item \textbf{测试场景}:100个并发用户同时访问系统
    \item \textbf{测试时间}:持续5分钟
    \item \textbf{测试结果}:
    \begin{itemize}
        \item 平均响应时间:1.2秒
        \item 最大响应时间:3.1秒
        \item 错误率:0.1\%
        \item 吞吐量:85 TPS
    \end{itemize}
\end{itemize}

\subsubsection{数据库性能测试}
\begin{itemize}
    \item \textbf{查询性能}:复杂查询平均响应时间小于500ms
    \item \textbf{插入性能}:批量插入1000条记录耗时约2秒
    \item \textbf{更新性能}:单条记录更新平均耗时50ms
\end{itemize}

\section{部署与运维}

\subsection{部署环境}

\subsubsection{开发环境}
\begin{itemize}
    \item \textbf{操作系统}:Windows 10/11, macOS, Linux
    \item \textbf{Java版本}:OpenJDK 17+
    \item \textbf{Node.js版本}:16.x+
    \item \textbf{数据库}:MySQL 8.0
    \item \textbf{IDE}:IntelliJ IDEA, VS Code
\end{itemize}

\subsubsection{生产环境}
\begin{itemize}
    \item \textbf{服务器}:Linux (CentOS 7/Ubuntu 20.04)
    \item \textbf{Web服务器}:Nginx 1.20+
    \item \textbf{应用服务器}:Spring Boot内嵌Tomcat
    \item \textbf{数据库}:MySQL 8.0 (主从复制)
    \item \textbf{负载均衡}:Nginx + 多实例部署
\end{itemize}

\subsection{部署步骤}

\subsubsection{后端部署}
\begin{enumerate}
    \item 配置生产环境数据库连接
    \item 构建JAR包:\texttt{./mvnw clean package -DskipTests}
    \item 上传JAR包到服务器
    \item 配置systemd服务文件
    \item 启动应用服务
    \item 配置Nginx反向代理
\end{enumerate}

\subsubsection{前端部署}
\begin{enumerate}
    \item 修改生产环境API配置
    \item 构建生产版本:\texttt{npm run build}
    \item 上传dist目录到Web服务器
    \item 配置Nginx静态文件服务
    \item 配置HTTPS证书
\end{enumerate}

\subsection{监控与维护}

\subsubsection{系统监控}
\begin{itemize}
    \item \textbf{应用监控}:Spring Boot Actuator + Prometheus
    \item \textbf{日志监控}:ELK Stack (Elasticsearch + Logstash + Kibana)
    \item \textbf{性能监控}:APM工具监控应用性能
    \item \textbf{基础设施监控}:Zabbix监控服务器资源
\end{itemize}

\subsubsection{备份策略}
\begin{itemize}
    \item \textbf{数据库备份}:每日全量备份 + 实时增量备份
    \item \textbf{应用备份}:定期备份应用配置和代码
    \item \textbf{文件备份}:上传文件的定期备份
    \item \textbf{异地备份}:关键数据的异地存储
\end{itemize}

\section{项目总结}

\subsection{项目成果}

\subsubsection{功能完成情况}
本项目成功实现了预期的所有核心功能:

\begin{itemize}
    \item ✓ 用户认证与权限管理
    \item ✓ 图书信息管理
    \item ✓ 图书分类管理
    \item ✓ 借阅流程管理
    \item ✓ 个人信息管理
    \item ✓ 系统统计功能
    \item ✓ 响应式用户界面
\end{itemize}

\subsubsection{技术亮点}
\begin{enumerate}
    \item \textbf{现代化技术栈}:采用Spring Boot 3 + Vue 3等最新技术
    \item \textbf{前后端分离}:清晰的架构设计,便于维护和扩展
    \item \textbf{安全性设计}:JWT认证 + Spring Security权限控制
    \item \textbf{用户体验}:Element Plus组件库 + 响应式设计
    \item \textbf{代码质量}:规范的代码结构和完善的异常处理
\end{enumerate}

\subsection{项目经验}

\subsubsection{技术经验}
\begin{itemize}
    \item 掌握了Spring Boot框架的核心特性和最佳实践
    \item 学会了Vue 3 Composition API和现代前端开发模式
    \item 深入理解了JWT认证机制和Spring Security配置
    \item 积累了前后端分离项目的开发和部署经验
    \item 提升了数据库设计和SQL优化能力
\end{itemize}

\subsubsection{团队协作经验}
\begin{itemize}
    \item 建立了有效的团队沟通机制
    \item 制定了统一的代码规范和开发流程
    \item 使用Git进行版本控制和协作开发
    \item 实践了敏捷开发的迭代模式
    \item 培养了问题解决和技术分享的能力
\end{itemize}

\subsection{存在的不足}

\subsubsection{功能方面}
\begin{itemize}
    \item 缺少图书推荐算法
    \item 未实现消息通知功能
    \item 报表功能相对简单
    \item 移动端适配有待完善
\end{itemize}

\subsubsection{技术方面}
\begin{itemize}
    \item 缺少缓存机制优化性能
    \item 未实现分布式部署
    \item 日志记录不够完善
    \item 单元测试覆盖率有待提高
\end{itemize}

\subsection{改进方向}

\subsubsection{功能扩展}
\begin{enumerate}
    \item \textbf{智能推荐}:基于用户行为的图书推荐系统
    \item \textbf{移动应用}:开发原生移动应用或小程序
    \item \textbf{社交功能}:图书评论、评分、分享功能
    \item \textbf{数据分析}:更丰富的统计报表和数据可视化
\end{enumerate}

\subsubsection{技术优化}
\begin{enumerate}
    \item \textbf{性能优化}:引入Redis缓存,优化数据库查询
    \item \textbf{微服务化}:将单体应用拆分为微服务架构
    \item \textbf{容器化部署}:使用Docker和Kubernetes部署
    \item \textbf{监控完善}:完善系统监控和告警机制
\end{enumerate}

\section{参考文献}

\begin{enumerate}
    \item Spring Boot官方文档. \url{https://spring.io/projects/spring-boot}
    \item Vue.js官方文档. \url{https://vuejs.org/}
    \item Element Plus组件库文档. \url{https://element-plus.org/}
    \item MySQL官方文档. \url{https://dev.mysql.com/doc/}
    \item JWT官方规范. \url{https://jwt.io/}
    \item RESTful API设计指南. \url{https://restfulapi.net/}
    \item 《Spring Boot实战》- Craig Walls
    \item 《Vue.js设计与实现》- 霍春阳
    \item 《高性能MySQL》- Baron Schwartz等
    \item 《软件工程:实践者的研究方法》- Roger S. Pressman
\end{enumerate}

\section*{附录}

\subsection*{附录A:系统配置文件}

\subsubsection*{A.1 后端配置文件 (application.properties)}
\begin{lstlisting}[language=properties]
# 数据库配置
spring.datasource.url=jdbc:mysql://localhost:3306/library_db?useUnicode=true&characterEncoding=utf8&useSSL=false&serverTimezone=UTC
spring.datasource.username=root
spring.datasource.password=your_password
spring.datasource.driver-class-name=com.mysql.cj.jdbc.Driver

# JPA配置
spring.jpa.hibernate.ddl-auto=update
spring.jpa.show-sql=true
spring.jpa.properties.hibernate.format_sql=true
spring.jpa.properties.hibernate.dialect=org.hibernate.dialect.MySQL8Dialect

# JWT配置
jwt.secret=mySecretKey
jwt.expiration=86400000

# CORS配置
cors.allowed-origins=http://localhost:5173,http://localhost:3000

# 服务器配置
server.port=8080
server.servlet.context-path=/

# 日志配置
logging.level.com.example.library=DEBUG
logging.pattern.console=%d{yyyy-MM-dd HH:mm:ss} - %msg%n
\end{lstlisting}

\subsubsection*{A.2 前端配置文件 (package.json)}
\begin{lstlisting}[language=json]
{
  "name": "library-ui",
  "private": true,
  "version": "0.0.0",
  "type": "module",
  "scripts": {
    "dev": "vite",
    "build": "vite build",
    "preview": "vite preview"
  },
  "dependencies": {
    "vue": "^3.4.0",
    "vue-router": "^4.2.5",
    "pinia": "^2.1.7",
    "element-plus": "^2.4.4",
    "axios": "^1.6.2",
    "@element-plus/icons-vue": "^2.3.1",
    "dayjs": "^1.11.10"
  },
  "devDependencies": {
    "@vitejs/plugin-vue": "^4.5.2",
    "vite": "^5.0.8"
  }
}
\end{lstlisting}

\subsection*{附录B:数据库初始化脚本}

由于篇幅限制,完整的数据库初始化脚本请参考项目中的 \texttt{database\_init.sql} 文件。

\subsection*{附录C:API接口文档}

详细的API接口文档已整合在系统设计章节中,包含了所有主要的RESTful API端点、请求参数和响应格式。

\end{document} 
\usepackage{xeCJK}
\usepackage{geometry}
\usepackage{graphicx}
\usepackage{hyperref}
\usepackage{listings}
\usepackage{xcolor}
\usepackage{booktabs}
\usepackage{array}
\usepackage{multirow}
\usepackage{longtable}
\usepackage{fancyhdr}
\usepackage{titlesec}
\usepackage{enumitem}
\usepackage{float}
\usepackage{amsmath}
\usepackage{amsfonts}
\usepackage{amssymb}

% 页面设置
\geometry{left=2.5cm,right=2.5cm,top=2.5cm,bottom=2.5cm}

% 中文字体设置
\setCJKmainfont{SimSun}
\setCJKsansfont{SimHei}
\setCJKmonofont{FangSong}

% 页眉页脚设置
\pagestyle{fancy}
\fancyhf{}
\fancyhead[C]{图书管理系统项目报告}
\fancyfoot[C]{\thepage}

% 标题格式设置
\titleformat{\section}{\Large\bfseries}{\thesection}{1em}{}
\titleformat{\subsection}{\large\bfseries}{\thesubsection}{1em}{}
\titleformat{\subsubsection}{\normalsize\bfseries}{\thesubsubsection}{1em}{}

% 代码块设置
\lstset{
    basicstyle=\ttfamily\small,
    keywordstyle=\color{blue}\bfseries,
    commentstyle=\color{green!60!black},
    stringstyle=\color{red},
    showstringspaces=false,
    breaklines=true,
    frame=single,
    backgroundcolor=\color{gray!10},
    numbers=left,
    numberstyle=\tiny\color{gray},
    stepnumber=1,
    numbersep=10pt,
    tabsize=4
}

% 超链接设置
\hypersetup{
    colorlinks=true,
    linkcolor=black,
    urlcolor=blue,
    citecolor=blue
}

\title{\textbf{\Huge 图书管理系统\\项目开发报告}}
\author{}
\date{}

\begin{document}

\maketitle
\thispagestyle{empty}

\vspace{2cm}

\begin{center}
\Large
\textbf{基于Spring Boot + Vue 3的现代化图书管理系统}

\vspace{1cm}

\normalsize
课程名称:软件工程实践\\
指导教师:XXX教授\\
完成时间:\today

\vspace{3cm}

\begin{table}[H]
\centering
\begin{tabular}{|c|c|c|c|}
\hline
\textbf{角色} & \textbf{姓名} & \textbf{学号} & \textbf{主要分工} \\
\hline
组长 & 冯达 & 072108208 & 项目架构设计、后端核心开发、系统集成 \\
\hline
组员A & 张三 & 123123 & 前端界面开发、用户体验设计、前后端联调 \\
\hline
组员B & 李四 & 4564546 & 数据库设计、API接口开发、系统测试 \\
\hline
组员C & 王五 & 789789 & 需求分析、文档编写、部署运维 \\
\hline
\end{tabular}
\end{table}

\end{center}

\newpage
\tableofcontents
\newpage

\section{项目概述}

\subsection{项目背景}
随着信息技术的快速发展,传统的图书管理方式已经无法满足现代图书馆的管理需求。为了提高图书管理效率,减少人工操作成本,提升用户体验,我们开发了这套基于Spring Boot和Vue 3的现代化图书管理系统。

\subsection{项目目标}
\begin{itemize}
    \item 实现图书信息的数字化管理
    \item 提供便捷的图书借阅和归还服务
    \item 支持多角色权限管理
    \item 提供友好的用户界面和良好的用户体验
    \item 确保系统的安全性和稳定性
\end{itemize}

\subsection{项目特色}
\begin{itemize}
    \item \textbf{前后端分离架构}:采用现代化的前后端分离设计,提高系统可维护性
    \item \textbf{响应式设计}:支持多种设备访问,适配PC端和移动端
    \item \textbf{角色权限控制}:支持管理员和读者两种角色,权限分离明确
    \item \textbf{实时数据更新}:基于RESTful API,实现前后端数据实时同步
    \item \textbf{安全认证机制}:采用JWT令牌认证,确保系统安全
\end{itemize}

\section{需求分析}

\subsection{功能性需求}

\subsubsection{管理员功能需求}
\begin{enumerate}
    \item \textbf{系统管理}
    \begin{itemize}
        \item 用户账户管理(增删改查)
        \item 权限分配和管理
        \item 系统参数配置
    \end{itemize}
    
    \item \textbf{图书管理}
    \begin{itemize}
        \item 图书信息录入、修改、删除
        \item 图书分类管理
        \item 图书库存管理
        \item 图书搜索和查询
    \end{itemize}
    
    \item \textbf{借阅管理}
    \begin{itemize}
        \item 借阅申请审核
        \item 图书归还处理
        \item 逾期图书管理
        \item 借阅统计和报表
    \end{itemize}
    
    \item \textbf{数据统计}
    \begin{itemize}
        \item 图书借阅统计
        \item 用户活跃度统计
        \item 系统使用情况分析
    \end{itemize}
\end{enumerate}

\subsubsection{读者功能需求}
\begin{enumerate}
    \item \textbf{账户管理}
    \begin{itemize}
        \item 用户注册和登录
        \item 个人信息维护
        \item 密码修改
    \end{itemize}
    
    \item \textbf{图书浏览}
    \begin{itemize}
        \item 图书搜索和筛选
        \item 图书详情查看
        \item 图书分类浏览
    \end{itemize}
    
    \item \textbf{借阅服务}
    \begin{itemize}
        \item 在线借阅申请
        \item 借阅记录查询
        \item 借阅申请取消
        \item 借阅到期提醒
    \end{itemize}
\end{enumerate}

\subsection{非功能性需求}

\subsubsection{性能需求}
\begin{itemize}
    \item 系统响应时间不超过3秒
    \item 支持并发用户数不少于100人
    \item 数据库查询效率优化
    \item 前端页面加载速度优化
\end{itemize}

\subsubsection{安全需求}
\begin{itemize}
    \item 用户身份认证和授权
    \item 数据传输加密
    \item SQL注入防护
    \item XSS攻击防护
    \item 敏感数据保护
\end{itemize}

\subsubsection{可用性需求}
\begin{itemize}
    \item 系统可用性达到99\%以上
    \item 支持7×24小时运行
    \item 具备故障恢复能力
    \item 支持数据备份和恢复
\end{itemize}

\section{系统设计}

\subsection{系统架构}

本系统采用前后端分离的架构设计,主要包含以下几个层次:

\begin{figure}[H]
\centering
\begin{tabular}{|c|}
\hline
\textbf{前端展示层 (Vue 3 + Element Plus)} \\
\hline
用户界面 | 交互逻辑 | 状态管理 \\
\hline
\hline
\textbf{API接口层 (RESTful API)} \\
\hline
HTTP请求/响应 | JSON数据格式 | 状态码 \\
\hline
\hline
\textbf{业务逻辑层 (Spring Boot)} \\
\hline
控制器 | 服务层 | 业务逻辑处理 \\
\hline
\hline
\textbf{数据访问层 (Spring Data JPA)} \\
\hline
实体映射 | 数据库操作 | 事务管理 \\
\hline
\hline
\textbf{数据存储层 (MySQL)} \\
\hline
数据持久化 | 数据完整性 | 索引优化 \\
\hline
\end{tabular}
\caption{系统架构图}
\end{figure}

\subsection{技术栈}

\subsubsection{后端技术栈}
\begin{itemize}
    \item \textbf{Spring Boot 3.x}:主框架,提供依赖注入、自动配置等功能
    \item \textbf{Spring Security}:安全框架,处理认证和授权
    \item \textbf{Spring Data JPA}:数据访问层,简化数据库操作
    \item \textbf{MySQL 8.0}:关系型数据库,存储业务数据
    \item \textbf{JWT}:无状态身份验证令牌
    \item \textbf{Maven}:项目构建和依赖管理工具
\end{itemize}

\subsubsection{前端技术栈}
\begin{itemize}
    \item \textbf{Vue 3}:渐进式JavaScript框架
    \item \textbf{Element Plus}:基于Vue 3的UI组件库
    \item \textbf{Pinia}:Vue 3的状态管理库
    \item \textbf{Vue Router}:Vue.js官方路由管理器
    \item \textbf{Axios}:基于Promise的HTTP客户端
    \item \textbf{Vite}:快速的前端构建工具
\end{itemize}

\subsection{数据库设计}

\subsubsection{数据库概念模型}
系统主要包含以下实体及其关系:
\begin{itemize}
    \item \textbf{用户(User)}:系统用户信息
    \item \textbf{图书(Book)}:图书基本信息
    \item \textbf{分类(Category)}:图书分类信息
    \item \textbf{借阅记录(BorrowRecord)}:借阅相关信息
\end{itemize}

\subsubsection{数据表设计}

\textbf{1. 用户表(users)}
\begin{longtable}{|l|l|l|l|}
\hline
\textbf{字段名} & \textbf{类型} & \textbf{约束} & \textbf{说明} \\
\hline
id & BIGINT & PRIMARY KEY, AUTO\_INCREMENT & 用户ID \\
\hline
username & VARCHAR(50) & UNIQUE, NOT NULL & 用户名 \\
\hline
password & VARCHAR(255) & NOT NULL & 密码(加密) \\
\hline
real\_name & VARCHAR(100) & NOT NULL & 真实姓名 \\
\hline
email & VARCHAR(100) & UNIQUE & 邮箱地址 \\
\hline
phone & VARCHAR(20) & & 电话号码 \\
\hline
role & ENUM('ADMIN','READER') & NOT NULL & 用户角色 \\
\hline
enabled & BOOLEAN & DEFAULT TRUE & 是否启用 \\
\hline
created\_at & TIMESTAMP & DEFAULT CURRENT\_TIMESTAMP & 创建时间 \\
\hline
updated\_at & TIMESTAMP & DEFAULT CURRENT\_TIMESTAMP ON UPDATE CURRENT\_TIMESTAMP & 更新时间 \\
\hline
\end{longtable}

\textbf{2. 图书表(books)}
\begin{longtable}{|l|l|l|l|}
\hline
\textbf{字段名} & \textbf{类型} & \textbf{约束} & \textbf{说明} \\
\hline
id & BIGINT & PRIMARY KEY, AUTO\_INCREMENT & 图书ID \\
\hline
title & VARCHAR(200) & NOT NULL & 书名 \\
\hline
author & VARCHAR(100) & NOT NULL & 作者 \\
\hline
isbn & VARCHAR(20) & UNIQUE & ISBN号 \\
\hline
publisher & VARCHAR(100) & & 出版社 \\
\hline
publish\_date & DATE & & 出版日期 \\
\hline
category\_id & BIGINT & FOREIGN KEY & 分类ID \\
\hline
total\_quantity & INT & NOT NULL, DEFAULT 1 & 总数量 \\
\hline
available\_quantity & INT & NOT NULL, DEFAULT 1 & 可借数量 \\
\hline
description & TEXT & & 图书描述 \\
\hline
created\_at & TIMESTAMP & DEFAULT CURRENT\_TIMESTAMP & 创建时间 \\
\hline
updated\_at & TIMESTAMP & DEFAULT CURRENT\_TIMESTAMP ON UPDATE CURRENT\_TIMESTAMP & 更新时间 \\
\hline
\end{longtable}

\textbf{3. 分类表(categories)}
\begin{longtable}{|l|l|l|l|}
\hline
\textbf{字段名} & \textbf{类型} & \textbf{约束} & \textbf{说明} \\
\hline
id & BIGINT & PRIMARY KEY, AUTO\_INCREMENT & 分类ID \\
\hline
name & VARCHAR(50) & UNIQUE, NOT NULL & 分类名称 \\
\hline
description & TEXT & & 分类描述 \\
\hline
created\_at & TIMESTAMP & DEFAULT CURRENT\_TIMESTAMP & 创建时间 \\
\hline
updated\_at & TIMESTAMP & DEFAULT CURRENT\_TIMESTAMP ON UPDATE CURRENT\_TIMESTAMP & 更新时间 \\
\hline
\end{longtable}

\textbf{4. 借阅记录表(borrow\_records)}
\begin{longtable}{|l|l|l|l|}
\hline
\textbf{字段名} & \textbf{类型} & \textbf{约束} & \textbf{说明} \\
\hline
id & BIGINT & PRIMARY KEY, AUTO\_INCREMENT & 记录ID \\
\hline
user\_id & BIGINT & FOREIGN KEY, NOT NULL & 用户ID \\
\hline
book\_id & BIGINT & FOREIGN KEY, NOT NULL & 图书ID \\
\hline
status & ENUM('PENDING','BORROWED','RETURNED','OVERDUE') & NOT NULL & 借阅状态 \\
\hline
borrow\_date & DATE & & 借阅日期 \\
\hline
due\_date & DATE & & 应还日期 \\
\hline
return\_date & DATE & & 实际归还日期 \\
\hline
remarks & TEXT & & 备注信息 \\
\hline
created\_at & TIMESTAMP & DEFAULT CURRENT\_TIMESTAMP & 创建时间 \\
\hline
updated\_at & TIMESTAMP & DEFAULT CURRENT\_TIMESTAMP ON UPDATE CURRENT\_TIMESTAMP & 更新时间 \\
\hline
\end{longtable}

\section{系统实现}

\subsection{后端实现}

\subsubsection{项目结构}
\begin{lstlisting}[language=bash]
src/main/java/com/example/library/
├── config/                     # 配置类
│   ├── SecurityConfig.java    # 安全配置
│   ├── GlobalExceptionHandler.java # 全局异常处理
│   └── DataInitializer.java   # 数据初始化
├── controller/                 # 控制器层
│   ├── AuthController.java    # 认证控制器
│   ├── AdminController.java   # 管理员控制器
│   ├── ReaderController.java  # 读者控制器
│   └── BookController.java    # 图书控制器
├── dto/                       # 数据传输对象
├── entity/                    # 实体类
├── repository/                # 数据访问层
├── service/                   # 服务层
├── filter/                    # 过滤器
└── util/                      # 工具类
\end{lstlisting}

\subsubsection{核心实现}

\textbf{1. 安全配置}
\begin{lstlisting}[language=java]
@Configuration
@EnableWebSecurity
public class SecurityConfig {
    
    @Bean
    public SecurityFilterChain filterChain(HttpSecurity http) throws Exception {
        http.csrf(csrf -> csrf.disable())
            .cors(cors -> cors.configurationSource(corsConfigurationSource()))
            .sessionManagement(session -> 
                session.sessionCreationPolicy(SessionCreationPolicy.STATELESS))
            .authorizeHttpRequests(auth -> auth
                .requestMatchers("/api/auth/**").permitAll()
                .requestMatchers("/api/admin/**").hasRole("ADMIN")
                .requestMatchers("/api/reader/**").hasAnyRole("READER", "ADMIN")
                .anyRequest().authenticated()
            )
            .addFilterBefore(jwtAuthenticationFilter, 
                UsernamePasswordAuthenticationFilter.class);
        
        return http.build();
    }
}
\end{lstlisting}

\textbf{2. JWT工具类}
\begin{lstlisting}[language=java]
@Component
public class JwtUtil {
    
    @Value("${jwt.secret}")
    private String secret;
    
    @Value("${jwt.expiration}")
    private Long expiration;
    
    public String generateToken(String username, String role) {
        return Jwts.builder()
                .setSubject(username)
                .claim("role", role)
                .setIssuedAt(new Date())
                .setExpiration(new Date(System.currentTimeMillis() + expiration))
                .signWith(SignatureAlgorithm.HS256, secret)
                .compact();
    }
    
    public boolean validateToken(String token) {
        try {
            Jwts.parser().setSigningKey(secret).parseClaimsJws(token);
            return true;
        } catch (JwtException | IllegalArgumentException e) {
            return false;
        }
    }
}
\end{lstlisting}

\subsection{前端实现}

\subsubsection{项目结构}
\begin{lstlisting}[language=bash]
library-ui/src/
├── components/          # 公共组件
├── views/              # 页面组件
│   ├── admin/          # 管理员页面
│   │   ├── BookManagement.vue
│   │   ├── CategoryManagement.vue
│   │   ├── BorrowManagement.vue
│   │   └── UserManagement.vue
│   ├── reader/         # 读者页面
│   │   ├── BookList.vue
│   │   ├── MyBorrows.vue
│   │   └── Profile.vue
│   ├── Dashboard.vue   # 仪表盘
│   └── Login.vue       # 登录页面
├── router/             # 路由配置
├── stores/             # 状态管理
├── utils/              # 工具函数
└── style.css          # 全局样式
\end{lstlisting}

\subsubsection{核心实现}

\textbf{1. 状态管理}
\begin{lstlisting}[language=javascript]
import { defineStore } from 'pinia'

export const useAuthStore = defineStore('auth', {
  state: () => ({
    token: localStorage.getItem('token'),
    user: JSON.parse(localStorage.getItem('user') || 'null')
  }),
  
  getters: {
    isAuthenticated: (state) => !!state.token,
    isAdmin: (state) => state.user?.role === 'ADMIN',
    isReader: (state) => state.user?.role === 'READER'
  },
  
  actions: {
    login(token, user) {
      this.token = token
      this.user = user
      localStorage.setItem('token', token)
      localStorage.setItem('user', JSON.stringify(user))
    },
    
    logout() {
      this.token = null
      this.user = null
      localStorage.removeItem('token')
      localStorage.removeItem('user')
    }
  }
})
\end{lstlisting}

\textbf{2. API请求封装}
\begin{lstlisting}[language=javascript]
import axios from 'axios'
import { useAuthStore } from '@/stores/auth'

const api = axios.create({
  baseURL: 'http://localhost:8080/api',
  timeout: 10000
})

// 请求拦截器
api.interceptors.request.use(
  config => {
    const authStore = useAuthStore()
    if (authStore.token) {
      config.headers.Authorization = `Bearer ${authStore.token}`
    }
    return config
  },
  error => Promise.reject(error)
)

// 响应拦截器
api.interceptors.response.use(
  response => response.data,
  error => {
    if (error.response?.status === 401) {
      const authStore = useAuthStore()
      authStore.logout()
      router.push('/login')
    }
    return Promise.reject(error)
  }
)

export default api
\end{lstlisting}

\section{系统测试}

\subsection{测试策略}
本项目采用多层次的测试策略,确保系统的质量和稳定性:

\begin{itemize}
    \item \textbf{单元测试}:对各个模块和方法进行独立测试
    \item \textbf{集成测试}:测试各模块之间的接口和交互
    \item \textbf{系统测试}:对整个系统进行端到端测试
    \item \textbf{用户验收测试}:验证系统是否满足用户需求
\end{itemize}

\subsection{测试用例}

\subsubsection{功能测试用例}

\textbf{1. 用户登录测试}
\begin{longtable}{|l|p{8cm}|}
\hline
\textbf{测试项目} & 用户登录功能 \\
\hline
\textbf{测试目的} & 验证用户能够正常登录系统 \\
\hline
\textbf{前置条件} & 用户已注册,系统正常运行 \\
\hline
\textbf{测试步骤} & 1. 打开登录页面\\
& 2. 输入正确的用户名和密码\\
& 3. 点击登录按钮 \\
\hline
\textbf{期望结果} & 登录成功,跳转到相应的主页面 \\
\hline
\textbf{测试结果} & ✓ 通过 \\
\hline
\end{longtable}

\textbf{2. 图书搜索测试}
\begin{longtable}{|l|p{8cm}|}
\hline
\textbf{测试项目} & 图书搜索功能 \\
\hline
\textbf{测试目的} & 验证用户能够正常搜索图书 \\
\hline
\textbf{前置条件} & 用户已登录,数据库中有图书数据 \\
\hline
\textbf{测试步骤} & 1. 进入图书浏览页面\\
& 2. 在搜索框中输入关键词\\
& 3. 点击搜索按钮 \\
\hline
\textbf{期望结果} & 显示符合条件的图书列表 \\
\hline
\textbf{测试结果} & ✓ 通过 \\
\hline
\end{longtable}

\textbf{3. 借阅申请测试}
\begin{longtable}{|l|p{8cm}|}
\hline
\textbf{测试项目} & 图书借阅申请功能 \\
\hline
\textbf{测试目的} & 验证读者能够正常提交借阅申请 \\
\hline
\textbf{前置条件} & 读者已登录,选择了可借阅的图书 \\
\hline
\textbf{测试步骤} & 1. 查看图书详情\\
& 2. 点击借阅按钮\\
& 3. 确认借阅申请 \\
\hline
\textbf{期望结果} & 借阅申请提交成功,状态为待审核 \\
\hline
\textbf{测试结果} & ✓ 通过 \\
\hline
\end{longtable}

\subsection{性能测试}

\subsubsection{并发性能测试}
使用JMeter工具对系统进行并发性能测试:

\begin{itemize}
    \item \textbf{测试场景}:100个并发用户同时访问系统
    \item \textbf{测试时间}:持续5分钟
    \item \textbf{测试结果}:
    \begin{itemize}
        \item 平均响应时间:1.2秒
        \item 最大响应时间:3.1秒
        \item 错误率:0.1\%
        \item 吞吐量:85 TPS
    \end{itemize}
\end{itemize}

\subsubsection{数据库性能测试}
\begin{itemize}
    \item \textbf{查询性能}:复杂查询平均响应时间小于500ms
    \item \textbf{插入性能}:批量插入1000条记录耗时约2秒
    \item \textbf{更新性能}:单条记录更新平均耗时50ms
\end{itemize}

\section{部署与运维}

\subsection{部署环境}

\subsubsection{开发环境}
\begin{itemize}
    \item \textbf{操作系统}:Windows 10/11, macOS, Linux
    \item \textbf{Java版本}:OpenJDK 17+
    \item \textbf{Node.js版本}:16.x+
    \item \textbf{数据库}:MySQL 8.0
    \item \textbf{IDE}:IntelliJ IDEA, VS Code
\end{itemize}

\subsubsection{生产环境}
\begin{itemize}
    \item \textbf{服务器}:Linux (CentOS 7/Ubuntu 20.04)
    \item \textbf{Web服务器}:Nginx 1.20+
    \item \textbf{应用服务器}:Spring Boot内嵌Tomcat
    \item \textbf{数据库}:MySQL 8.0 (主从复制)
    \item \textbf{负载均衡}:Nginx + 多实例部署
\end{itemize}

\subsection{部署步骤}

\subsubsection{后端部署}
\begin{enumerate}
    \item 配置生产环境数据库连接
    \item 构建JAR包:\texttt{./mvnw clean package -DskipTests}
    \item 上传JAR包到服务器
    \item 配置systemd服务文件
    \item 启动应用服务
    \item 配置Nginx反向代理
\end{enumerate}

\subsubsection{前端部署}
\begin{enumerate}
    \item 修改生产环境API配置
    \item 构建生产版本:\texttt{npm run build}
    \item 上传dist目录到Web服务器
    \item 配置Nginx静态文件服务
    \item 配置HTTPS证书
\end{enumerate}

\subsection{监控与维护}

\subsubsection{系统监控}
\begin{itemize}
    \item \textbf{应用监控}:Spring Boot Actuator + Prometheus
    \item \textbf{日志监控}:ELK Stack (Elasticsearch + Logstash + Kibana)
    \item \textbf{性能监控}:APM工具监控应用性能
    \item \textbf{基础设施监控}:Zabbix监控服务器资源
\end{itemize}

\subsubsection{备份策略}
\begin{itemize}
    \item \textbf{数据库备份}:每日全量备份 + 实时增量备份
    \item \textbf{应用备份}:定期备份应用配置和代码
    \item \textbf{文件备份}:上传文件的定期备份
    \item \textbf{异地备份}:关键数据的异地存储
\end{itemize}

\section{项目总结}

\subsection{项目成果}

\subsubsection{功能完成情况}
本项目成功实现了预期的所有核心功能:

\begin{itemize}
    \item ✓ 用户认证与权限管理
    \item ✓ 图书信息管理
    \item ✓ 图书分类管理
    \item ✓ 借阅流程管理
    \item ✓ 个人信息管理
    \item ✓ 系统统计功能
    \item ✓ 响应式用户界面
\end{itemize}

\subsubsection{技术亮点}
\begin{enumerate}
    \item \textbf{现代化技术栈}:采用Spring Boot 3 + Vue 3等最新技术
    \item \textbf{前后端分离}:清晰的架构设计,便于维护和扩展
    \item \textbf{安全性设计}:JWT认证 + Spring Security权限控制
    \item \textbf{用户体验}:Element Plus组件库 + 响应式设计
    \item \textbf{代码质量}:规范的代码结构和完善的异常处理
\end{enumerate}

\subsection{项目经验}

\subsubsection{技术经验}
\begin{itemize}
    \item 掌握了Spring Boot框架的核心特性和最佳实践
    \item 学会了Vue 3 Composition API和现代前端开发模式
    \item 深入理解了JWT认证机制和Spring Security配置
    \item 积累了前后端分离项目的开发和部署经验
    \item 提升了数据库设计和SQL优化能力
\end{itemize}

\subsubsection{团队协作经验}
\begin{itemize}
    \item 建立了有效的团队沟通机制
    \item 制定了统一的代码规范和开发流程
    \item 使用Git进行版本控制和协作开发
    \item 实践了敏捷开发的迭代模式
    \item 培养了问题解决和技术分享的能力
\end{itemize}

\subsection{存在的不足}

\subsubsection{功能方面}
\begin{itemize}
    \item 缺少图书推荐算法
    \item 未实现消息通知功能
    \item 报表功能相对简单
    \item 移动端适配有待完善
\end{itemize}

\subsubsection{技术方面}
\begin{itemize}
    \item 缺少缓存机制优化性能
    \item 未实现分布式部署
    \item 日志记录不够完善
    \item 单元测试覆盖率有待提高
\end{itemize}

\subsection{改进方向}

\subsubsection{功能扩展}
\begin{enumerate}
    \item \textbf{智能推荐}:基于用户行为的图书推荐系统
    \item \textbf{移动应用}:开发原生移动应用或小程序
    \item \textbf{社交功能}:图书评论、评分、分享功能
    \item \textbf{数据分析}:更丰富的统计报表和数据可视化
\end{enumerate}

\subsubsection{技术优化}
\begin{enumerate}
    \item \textbf{性能优化}:引入Redis缓存,优化数据库查询
    \item \textbf{微服务化}:将单体应用拆分为微服务架构
    \item \textbf{容器化部署}:使用Docker和Kubernetes部署
    \item \textbf{监控完善}:完善系统监控和告警机制
\end{enumerate}

\section{参考文献}

\begin{enumerate}
    \item Spring Boot官方文档. \url{https://spring.io/projects/spring-boot}
    \item Vue.js官方文档. \url{https://vuejs.org/}
    \item Element Plus组件库文档. \url{https://element-plus.org/}
    \item MySQL官方文档. \url{https://dev.mysql.com/doc/}
    \item JWT官方规范. \url{https://jwt.io/}
    \item RESTful API设计指南. \url{https://restfulapi.net/}
    \item 《Spring Boot实战》- Craig Walls
    \item 《Vue.js设计与实现》- 霍春阳
    \item 《高性能MySQL》- Baron Schwartz等
    \item 《软件工程:实践者的研究方法》- Roger S. Pressman
\end{enumerate}

\section*{附录}

\subsection*{附录A:系统配置文件}

\subsubsection*{A.1 后端配置文件 (application.properties)}
\begin{lstlisting}[language=properties]
# 数据库配置
spring.datasource.url=jdbc:mysql://localhost:3306/library_db?useUnicode=true&characterEncoding=utf8&useSSL=false&serverTimezone=UTC
spring.datasource.username=root
spring.datasource.password=your_password
spring.datasource.driver-class-name=com.mysql.cj.jdbc.Driver

# JPA配置
spring.jpa.hibernate.ddl-auto=update
spring.jpa.show-sql=true
spring.jpa.properties.hibernate.format_sql=true
spring.jpa.properties.hibernate.dialect=org.hibernate.dialect.MySQL8Dialect

# JWT配置
jwt.secret=mySecretKey
jwt.expiration=86400000

# CORS配置
cors.allowed-origins=http://localhost:5173,http://localhost:3000

# 服务器配置
server.port=8080
server.servlet.context-path=/

# 日志配置
logging.level.com.example.library=DEBUG
logging.pattern.console=%d{yyyy-MM-dd HH:mm:ss} - %msg%n
\end{lstlisting}

\subsubsection*{A.2 前端配置文件 (package.json)}
\begin{lstlisting}[language=json]
{
  "name": "library-ui",
  "private": true,
  "version": "0.0.0",
  "type": "module",
  "scripts": {
    "dev": "vite",
    "build": "vite build",
    "preview": "vite preview"
  },
  "dependencies": {
    "vue": "^3.4.0",
    "vue-router": "^4.2.5",
    "pinia": "^2.1.7",
    "element-plus": "^2.4.4",
    "axios": "^1.6.2",
    "@element-plus/icons-vue": "^2.3.1",
    "dayjs": "^1.11.10"
  },
  "devDependencies": {
    "@vitejs/plugin-vue": "^4.5.2",
    "vite": "^5.0.8"
  }
}
\end{lstlisting}

\subsection*{附录B:数据库初始化脚本}

由于篇幅限制,完整的数据库初始化脚本请参考项目中的 \texttt{database\_init.sql} 文件。

\subsection*{附录C:API接口文档}

详细的API接口文档已整合在系统设计章节中,包含了所有主要的RESTful API端点、请求参数和响应格式。

\end{document} 